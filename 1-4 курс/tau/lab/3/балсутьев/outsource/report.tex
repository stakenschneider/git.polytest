\documentclass[12pt,a4paper]{article}
\usepackage[left=3cm, right=1cm, top=2cm, bottom=2cm]{geometry}
\usepackage[utf8]{inputenc}
\usepackage[russian]{babel}
\usepackage[OT1]{fontenc}
\usepackage{amsmath}
\usepackage{amsfonts}
\usepackage{amssymb}
\usepackage{csquotes}
\usepackage{graphicx}
\usepackage{float}
\usepackage{listings}
\graphicspath{{image/}}
\bibliographystyle{plain}

\usepackage{placeins}

\lstdefinestyle{base_listing}{
  extendedchars     = {true},
  inputencoding     = {utf8}, 
  basicstyle        = {\ttfamily \scriptsize},
  keywordstyle      = {\rmfamily \bfseries},
  commentstyle      = {\rmfamily \itshape},
  tabsize           = {2},
  flexiblecolumns   = {false},
  frame             = {single},
  showstringspaces  = {false},
  breaklines        = {true}, 
  breakatwhitespace = {true}
}

\lstdefinelanguage{LLVM-asm}
{
  morekeywords = {
    load, store, malloc, alloca, free, getelementptr,
    add, sub, insertvalue, extractvalue, icmp, call,
    define, void, global
  },
  sensitive   = false,
  morecomment = [l]{;}
}

\lstdefinestyle{crs_llvm}{
  style    = {base_listing},
  language = {LLVM-asm}
}

\lstdefinestyle{crs_cpp}{
  style    = {base_listing},
  language = {C++}
}

\lstdefinestyle{crs_bash}{
  style    = {base_listing},
  language = {bash}
}


\begin{document}
	\begin{titlepage}
		\begin{center}
			Санкт-Петербургский политехнический университет Петра Великого\\
			Институт компьютерных наук и технологий\\
			Кафедра компьютерных систем и программных технологий\\
			\vspace{6cm}
			Отчет по лабораторной работе №3\\
			\vspace{0.2cm}	
			по дисциплине "Основы теории управления"\\
			\vspace{0.5cm}	
			\Large
			Свойства объекта. Переход между формами ВСВ
			\small
		\end{center}
		\vspace{5cm}
		\begin{flushright}
			Выполнил: студент группы 43501/4	Соболь В.\\
			\vspace{0.5cm}			
			Преподаватель: Нестеров С.А.\\
		\end{flushright}
		\vspace{7cm}
		\begin{center}
			Санкт-Петербург\\
			2018 г.\\
		\end{center}
	\end{titlepage}
	\newpage
\def\contentsname{Содержание}


% Contents
%\tableofcontents
%\clearpage


\section{Цель работы}

Анализ свойств заданного объекта. Построение матриц для перехода между формами представления ВСВ.

\section{Программа работы}

\begin{enumerate}
	\item Проанализировать свойства заданного объекта:
\begin{itemize}
\item управляемость
\item наблюдаемость
\item устойчивость
\item минимальность
\item минимально фазовость
\end{itemize}
\item Рассчитать матрицы перехода

\end{enumerate}

\section{Индивидуальное задание}

$x''+2x'+0.75x=0.75u$, \\

\section{Ход работы}
\subsection{Свойства объекта}

\subsubsection{Управляемость}

Проверим управляемость системы по критерию Калмана:

\begin{equation*}
\text{$A=\begin{bmatrix}
0 & 1 \\ -0.75 & -2
\end{bmatrix} B=\begin{bmatrix}
0 \\ 1
\end{bmatrix}$} \quad
\text{$
U = \begin{bmatrix}
B & AB
\end{bmatrix} = \begin{bmatrix}
0 & 1\\ 1 & -2
\end{bmatrix} 
$} \quad
\text{$detU=-1\neq 0$}
\end{equation*}

Определитель одной из матриц управляемости не нулевой, что означает, что система полностью управляема.

\subsubsection{Наблюдаемость}

Проверим наблюдаемость системы по критерию Калмана:

\begin{equation*}
\text{$A=\begin{bmatrix}
0 & -0.75 \\ 1 & -2
\end{bmatrix} C=\begin{bmatrix}
0 & 1
\end{bmatrix}$} \quad
\end{equation*}

\begin{equation*}
\text{$N=[C^T,A^TC^T]=[\begin{bmatrix} 0 \\ 1 \end{bmatrix},\begin{bmatrix}   0  &  1 \\
   -0.75 &  -2 \end{bmatrix}\begin{bmatrix} 0 \\ 1 \end{bmatrix}]=\begin{bmatrix} 0 & 1 \\ 1 & -2 \end{bmatrix}$}
\end{equation*}

\begin{equation*}
\text{$detN=-1\neq 0$}
\end{equation*}

Определитель одной из матриц наблюдаемости не нулевой, что означает, что система полностью наблюдаема.

\subsubsection{Устойчивость}

По теореме Ляпунова система является устойчивой тогда, когда вещественные части полюсов её передаточной функции отрицательны. В нашем случае полюса передаточной функции равны $p_1=-1.5, p_2=-0.5$, что означает, что система устойчива, так как все корни лежат в левой полуплоскости. 

\subsubsection{Минимальность}

Система минимальна, так как её порядок понизить нельзя.

\subsection{Минимально фазовость}

Звено является минимально фазовым, так как все корни лежат в левой полуплоскости. 

\subsection{Преобразования форм}

\subsubsection{Матрицы управляемости}

Матрица управляемости находится как блочная матрица, где первый столбец равен матрице $B$, а второй столбец равен произведению $AB$:

\begin{center}
$U=[B, AB]$
\end{center}

Матрицы управляемости нормальной формы управления (НФУ):

\begin{equation*}
\text{$U= \begin{bmatrix}
0 & 1\\ 1 & -2
\end{bmatrix}$}
\text{$, U^{-1}=\begin{bmatrix} 2 & 1 \\ 1 & 0 \end{bmatrix}$}
\end{equation*}

Матрицы управляемости нормальной формы наблюдения (НФН):

\begin{equation*}
\text{$U=\begin{bmatrix} 0.75 &         0 \\
         0  &  0.75 \end{bmatrix}$}
\quad
\text{$U^{-1}=\begin{bmatrix} 1.3333   &      0 \\
         0  &  1.3333 \end{bmatrix}$}
\end{equation*}

Матрицы управляемости канонической формы (КФ):

\begin{equation*}
\text{$U=\begin{bmatrix}  -0.75  &  1.125 \\
    0.75  & -0.375 \end{bmatrix}$}
\quad
\text{$U^{-1}=\begin{bmatrix} 0.6667 &   2 \\
    1.3333  &  1.3333 \end{bmatrix}$}
\end{equation*}

\subsubsection{Матрицы преобразования}

Матрица преобразования высчитывается по формуле:

\begin{center}
$P=U_{*}U^{-1}$
\end{center}

\begin{itemize}
	\item Матрица преобразования из НФУ
	\begin{itemize}
	\item НФУ в НФН
	\begin{equation*}
	\text{$P=U_{*}U^{-1}=
	\begin{bmatrix} 0.75 &         0 \\
         0  &  0.75 \end{bmatrix}
	\begin{bmatrix} 2 & 1 \\ 1 & 0 \end{bmatrix}
	=\begin{bmatrix}1.5 &   0.75 \\
    0.75 &         0 \end{bmatrix}$}
	\end{equation*}
	
	Проверим корректность полученной матрицы преобразования $P$. Для этого получим матрицу $B_{*}$ через матрицу $B$. 
	
	\begin{equation*}
	\text{$B_{*}=PB$}
	\Longrightarrow
	\text{$B_{*}
	=\begin{bmatrix}1.5 &   0.75 \\
    0.75 &         0 \end{bmatrix}
	\begin{bmatrix} 0 \\ 1 \end{bmatrix}
	=\begin{bmatrix} 0.75 \\ 0 \end{bmatrix}$}
	\end{equation*}
	
	\item НФУ в КФ:
	
	\begin{equation*}
	\text{$P=U_{*}U^{-1}=
	\begin{bmatrix}  -0.75  &  1.125 \\
    0.75  & -0.375 \end{bmatrix}
	\begin{bmatrix} 2 & 1 \\ 1 & 0 \end{bmatrix}
	=\begin{bmatrix} -0.375 &   -0.75 \\
    1.125 &    0.75 \end{bmatrix}$}
	\end{equation*}
	
	Проверим корректность полученной матрицы преобразования $P$. Для этого получим матрицу $B_{*}$ через матрицу $B$. 
	
	\begin{equation*}
	\text{$B_{*}=PB$}
	\Longrightarrow
	\text{$B_{*}=
	\begin{bmatrix} -0.375 &   -0.75 \\
    1.125 &    0.75 \end{bmatrix}
	\begin{bmatrix} 0 \\ 1 \end{bmatrix}
	=\begin{bmatrix} -0.75 \\ 0.75 \end{bmatrix}$}
	\end{equation*}
	
	\end{itemize}
	\item Матрица преобразования из НФН:
	\begin{itemize}

	\item НФН в НФУ
		
	\begin{equation*}
	\text{$P=U_{*}U^{-1}=
	\begin{bmatrix}
0 & 1\\ 1 & -2
\end{bmatrix}
\begin{bmatrix} 1.3333   &      0 \\
         0  &  1.3333 \end{bmatrix}
=\begin{bmatrix}   0  &  1.3333 \\
    1.3333  & -2.6667 \end{bmatrix}$}
	\end{equation*}
	
	Проверим корректность полученной матрицы преобразования $P$. Для этого получим матрицу $B_{*}$ через матрицу $B$.
	
	\begin{equation*}
	\text{$B_{*}=PB$}
	\Longrightarrow
	\text{$B_{*}=
	\begin{bmatrix}   0  &  1.3333 \\
    1.3333  & -2.6667 \end{bmatrix}
   \begin{bmatrix} 0.75 \\ 0 \end{bmatrix}
   =\begin{bmatrix} 0 \\ 1 \end{bmatrix}$}
	\end{equation*}
	
	
	
	\item НФН в КФ:
	
	\begin{equation*}
	\text{$P=U_{*}U^{-1}=
	\begin{bmatrix}  -0.75  &  1.125 \\
    0.75  & -0.375 \end{bmatrix}
\begin{bmatrix} 1.3333   &      0 \\
         0  &  1.3333 \end{bmatrix}
=\begin{bmatrix}   -1 &    1.5 \\
    1  & -0.5 \end{bmatrix}$}
	\end{equation*}
	
	Проверим корректность полученной матрицы преобразования $P$. Для этого получим матрицу $B_{*}$ через матрицу $B$.
	
	\begin{equation*}
	\text{$B_{*}=PB$}
	\Longrightarrow
	\text{$B_{*}=
	\begin{bmatrix}   -1 &    1.5 \\
    1  & -0.5 \end{bmatrix}
\begin{bmatrix} 0.75 \\ 0 \end{bmatrix}
   =\begin{bmatrix} -0.75 \\ 0.75 \end{bmatrix}$}
	\end{equation*}
	

	\end{itemize}	
	
	\item Матрица преобразования из КФ:
	\begin{itemize}
	\item КФ в НФУ

\begin{equation*}
	\text{$P=U_{*}U^{-1}=
	\begin{bmatrix}
0 & 1\\ 1 & -2
\end{bmatrix}
\begin{bmatrix} 0.6667 &   2 \\
    1.3333  &  1.3333 \end{bmatrix}
=\begin{bmatrix}1.3333  &  1.3333 \\
   -2  & -0.6667\end{bmatrix}$}
	\end{equation*}
	
	Проверим корректность полученной матрицы преобразования $P$. Для этого получим матрицу $B_{*}$ через матрицу $B$.
	
	\begin{equation*}
	\text{$B_{*}=PB$}
	\Longrightarrow
	\text{$B_{*}=
	\begin{bmatrix}1.3333  &  1.3333 \\
   -2  & -0.6667\end{bmatrix}
   \begin{bmatrix} -0.75 \\ 0.75 \end{bmatrix}
   =\begin{bmatrix} 0 \\ 1 \end{bmatrix}$}
	\end{equation*}
	

	
	\item КФ в НФН:
	
	\begin{equation*}
	\text{$P=U_{*}U^{-1}=
	\begin{bmatrix} 0.75 &         0 \\
         0  &  0.75 \end{bmatrix}
	\begin{bmatrix} 0.6667 &   2 \\
    1.3333  &  1.3333 \end{bmatrix}
=\begin{bmatrix}0.5 &  1.5 \\
    1  &  1\end{bmatrix}$}
	\end{equation*}
	
	Проверим корректность полученной матрицы преобразования $P$. Для этого получим матрицу $B_{*}$ через матрицу $B$.
	
	\begin{equation*}
	\text{$B_{*}=PB$}
	\Longrightarrow
	\text{$B_{*}=
\begin{bmatrix}0.5 &  1.5 \\
    1  &  1\end{bmatrix}
         \begin{bmatrix} -0.75 \\ 0.75 \end{bmatrix}
         =\begin{bmatrix} 0.75 \\ 0 \end{bmatrix}$}
	\end{equation*}
		
	\end{itemize}
\end{itemize}


\section{Вывод}

В ходе работы проанализированы свойства объекта и получены матрицы для переходов между
формами представления модели ВСВ. Также проверена корректность всех матриц перехода.
 
\end{document}