\documentclass[14pt,a4paper,report]{report}
\usepackage[a4paper, mag=1000, left=2.5cm, right=1cm, top=2cm, bottom=2cm, headsep=0.7cm, footskip=1cm]{geometry}
\usepackage[utf8]{inputenc}
\usepackage[english,russian]{babel}
\usepackage{indentfirst}
\usepackage[dvipsnames]{xcolor}
\usepackage[colorlinks]{hyperref}
\usepackage{listings} 
\usepackage{fancyhdr}
\usepackage{caption}
\usepackage{amsmath}
\usepackage{latexsym}
\usepackage{graphicx}
\usepackage{amsmath}
\hypersetup{
	colorlinks = true,
	linkcolor  = black
}

\usepackage{titlesec}
\titleformat{\chapter}
{\Large\bfseries} % format
{}                % label
{0pt}             % sep
{\huge}           % before-code


\DeclareCaptionFont{white}{\color{white}} 

% Listing description
\usepackage{listings} 
\DeclareCaptionFormat{listing}{\colorbox{gray}{\parbox{\textwidth}{#1#2#3}}}
\captionsetup[lstlisting]{format=listing,labelfont=white,textfont=white}
\lstset{ 
	% Listing settings
	inputencoding = utf8,			
	extendedchars = \true, 
	keepspaces = true, 			  	 % Поддержка кириллицы и пробелов в комментариях
	language = Matlab,            	 	 % Язык программирования (для подсветки)
	basicstyle = \small\sffamily, 	 % Размер и начертание шрифта для подсветки кода
	numbers = left,               	 % Где поставить нумерацию строк (слева\справа)
	numberstyle = \tiny,          	 % Размер шрифта для номеров строк
	stepnumber = 1,               	 % Размер шага между двумя номерами строк
	numbersep = 5pt,              	 % Как далеко отстоят номера строк от подсвечиваемого кода
	backgroundcolor = \color{white}, % Цвет фона подсветки - используем \usepackage{color}
	showspaces = false,           	 % Показывать или нет пробелы специальными отступами
	showstringspaces = false,    	 % Показывать или нет пробелы в строках
	showtabs = false,           	 % Показывать или нет табуляцию в строках
	frame = single,              	 % Рисовать рамку вокруг кода
	tabsize = 2,                  	 % Размер табуляции по умолчанию равен 2 пробелам
	captionpos = t,             	 % Позиция заголовка вверху [t] или внизу [b] 
	breaklines = true,           	 % Автоматически переносить строки (да\нет)
	breakatwhitespace = false,   	 % Переносить строки только если есть пробел
	escapeinside = {\%*}{*)}      	 % Если нужно добавить комментарии в коде
}

\begin{document}

\def\contentsname{Содержание}

% Titlepage
\begin{titlepage}
	\begin{center}
		\textsc{Санкт-Петербургский Политехнический 
			Университет Петра Великого\\[5mm]
			Кафедра компьютерных систем и программных технологий}
		
		\vfill
		
		\textbf{Отчёт по лабораторной работе №3\\[3mm]
			Курс: «Теория автоматического управления»\\[3mm]
			Тема: «Оптимизация качества системы»\\[35mm]
			}
	\end{center}
	
	\hfill
	\begin{minipage}{.5\textwidth}
		Выполнил студент:\\[2mm] 
		Бояркин Никита Сергеевич\\
		Группа: 43501/3\\[5mm]
		
		Проверил:\\[2mm] 
		Нестеров Сергей Александрович
	\end{minipage}
	\vfill
	\begin{center}
		Санкт-Петербург\\ \the\year\ г.
	\end{center}
\end{titlepage}

% Contents
\tableofcontents
\clearpage

\chapter{Лабораторная работа №3}

\section{Цель работы}

Научиться определять оптимальные критерии качества для замкнутой системы.

\section{Программа работы}

\begin{itemize}
	\item Определить область устойчивости
	\item Определить величину статической ошибки.
	\item Получить корневые критерии качества.
	\item Получить частотные критерии качества.
	\item Получить интегральные критерии качества.
	\item Промоделировать процессы в системе при оптимальных параметрах при наличии шума и без.
\end{itemize}

\section{Индивидуальное задание}

$
\\
y''+25y'=5u'+25u, y(0)=0, y'(0)=0, u=1(t)\\\\
W(p)=\frac{y}{u}=\frac{5p+25}{p^2+25p}
$

\section{Ход работы}

\subsection{Исходные данные замкнутой системы}

Структура исследуемой системы с добавлением изодромного звена и шума:

\begin{figure}[h!]
	\centering
	\includegraphics[scale = 0.70]{images/schema.png}
	\caption{Структурная схема системы}
	\label{image:1}
\end{figure}

\clearpage

Определим передаточную функцию разомкнутой системы:

\begin{center}
$W_p=\frac{B(p)}{C(p)}=\frac{k(Tp+1)}{p}\frac{5p+25}{p^2+25p}=\frac{5k(Tp^2+(5T+1)p+5)}{p(p^2+25p)}$
\end{center}

Определим характеристический полином замкнутой системы:

\begin{center}
$D(p)=B(p)+C(p)=p(p^2+25p)+5k(Tp^2+(5T+1)p+5)=p^3+5(kT+5)p^2+5k(5T+1)p+25k$
\end{center}

Определим передаточную функцию замкнутой системы:

\begin{center}
$W_3=\frac{B(p)}{B(p)+C(p)}=\frac{B(p)}{D(p)}=\frac{5k(Tp^2+(5T+1)p+5)}{p^3+5(kT+5)p^2+5k(5T+1)p+25k}$
\end{center}

\subsection{Определение области устойчивости}

Для выполнения необходимого условия устойчивости системы необходимо, чтобы коэффициенты характеристического полинома были положительны. Для этого должны выполняться следующие условия:

\begin{equation*}
\begin{cases}
	\text{$5(kT+5)>0$} \\
	\text{$5k(5T+1)>0$} \\
	\text{$25k>0$}
\end{cases}
\Longrightarrow
\begin{cases}
	\text{$kT>-5$} \\
	\text{$5kT>k$} \\
	\text{$k>0$}
\end{cases}
\end{equation*}

Так $T$ постоянная времени, то она не может быть отрицательной, поэтому результирующие условия устойчивости:

\begin{equation*}
\begin{cases}
	\text{$T>0$} \\
	\text{$k>0$}
\end{cases}
\end{equation*}

Для определения достаточного условия устойчивости воспользуемся критерием Гурвица для системы третьего порядка:

\begin{center}
$a_2a_1-a_3a_0>0$ \\
$5(kT+5)5k(5T+1)-25k>0$ \\
$(kT+5)(5T+1)-1>0$ \\
$5kT^2+(25+k)T+4>0$ \\
\end{center}

Из неравенства очевидно, что для всех $k$ и $T$, удовлетворяющих достаточному условию, необходимое условие также соблюдается.

\subsection{Статическая ошибка}

Для данной системы статическая ошибка вычисляется следующим образом:

\begin{center}
	$e=lim_{t\rightarrow\infty}\frac{1(t)}{1+W_p(t)}$
\end{center}

Так как система является астатической, то при $t\rightarrow\infty$ ошибка будет стремиться к нулю независимо от входного сигнала.

\subsection{Корневые критерии качества}

Данная группа критериев применяется для оценки качества системы по корням характеристического полинома:

\begin{center}
$D(p)=p^3+5(kT+5)p^2+5k(5T+1)p+25k$
\end{center}

\textbf{Оценка быстродействия} может производиться на основе величины:

\begin{center}
$\Omega=\sqrt[n]{|p_1\cdot...\cdot p_n|}$
\end{center}

Для данной системы существует три корня, которые легко находятся по теореме Виета:

\begin{center}
$\Omega=\sqrt[3]{|p_1\cdot p_2\cdot p_3|}=\sqrt[3]{|-a_3/a_0|}=\sqrt[3]{25k}$
\end{center}

\textbf{Степень устойчивости} системы определяется как абсолютное значение реальной части корней, ближайших к мнимой оси корня (к нулю):

\begin{center}
$realPart = min(|Re(p_1)|, |Re(p_2)|, |Re(p_3)|)$
\end{center}

Таким образом, для получения оптимальных параметров $k$ и $T$, значение $realPart$ нужно минимизировать.

\textbf{Колебательность системы} определяется мнимыми частями корней. Для нулевой колебательности все мнимые части коренй должны быть равны нулю:

\begin{center}
$imaginePart = (Im(p_1)=0\quad and\quad Im(p_2)=0\quad and\quad Im(p_3)=0)$
\end{center}

Таким образом, для получения оптимальных параметров $k$ и $T$, значение $imaginePart$ должно быть $True$.

\subsection{Частотные критерии качества}

Для оценки качества системы по частотным критериям представим передаточную функцию в частотном виде:

\begin{center}
$W_3(j\omega)=Re(\omega)+Im(\omega)j$  \linebreak \linebreak
$Re(\omega)=\frac{5k((5kT^2+20T-1)\omega^4+5(25kT+k-25)\omega^2+125k)}{Zn(\omega)}$ \linebreak \linebreak
$Im(\omega)=-\frac{5k(T\omega^5+5(25T+4)\omega^3)}{Zn(\omega)}$ \linebreak \linebreak
$Zn(\omega)=25(5k-(kT+5)\omega^2)^2+(5k(5T+1)\omega-\omega^3)^2$ \linebreak \linebreak
$A(\omega)=\sqrt{Re^2(\omega)+Im^2(\omega)}$ \linebreak \linebreak
$L(\omega)=20lg(A(\omega))$
\end{center}

\textbf{Показатель колебательности} определяется как отношение максимального модуля АЧХ к его значению при нулевой частоте:

\begin{center}
$\theta=\frac{max(A(\omega))}{A(0)}$
\end{center}

Так как значение АЧХ при нулевой частоте равно единице для любых значений $k$ и $T$:

\begin{center}
$\theta=max(A(\omega))$
\end{center}

Таким образом, для получения оптимальных параметров $k$ и $T$, значение $\theta$ нужно минимизировать. Однако, стоит отметить, что ниже единицы показатель колебательности быть не может, потому что при нулевой частоте он всегда равен единице (идеальный показатель колебательности).

\textbf{Запас устойчивости по амплитуде} определяется следующим образом:

\begin{center}
$C(\theta)=\frac{\theta^2}{\theta^2-1}$
\end{center}

Тогда идеальный запас устойчивости по амплитуде равен бесконечности.

\textbf{Запас устойчивости по фазе} определяется следующим образом:

\begin{center}
$\mu(\theta)=arccos(1-\frac{\theta^2}{2})$
\end{center}

Тогда идеальный запас устойчивости по фазе равен $\pi/3$.

\subsection{Интегральные критерии качества}

Воспользуемся квадратичным критерием качества:

\begin{center}
$I=\int_{0}^{\infty}x^2(t)dt$
\end{center}

Для данной системы $x^2(t)=(h(t)-1(t))^2$, где $h(t)$ - переходная характеристика замкнутой системы, а $1(t)$ - входное воздействие:

\begin{center}
$I=\int_{0}^{\infty}(h(t)-1(t))^2dt$
\end{center}

Таким образом, для получения оптимальных параметров $k$ и $T$, значение $I$  нужно минимизировать.

\subsection{Получение оптимальных критериев качества}

Воспользуемся средой \emph{Matlab} для поиска оптимальных параметров $k$ и $T$. Все вышеперечисленные условия должны по возможности выполняться.

\lstinputlisting{listings/script.m}

Данный скрипт находит оптимальные значения $k$ и $T$, соответствующие вышеперечисленным условиям, после чего рассчитывает критерии качества, рисует графики переходной характеристики с шумом и без.

В ходе исследования было выяснено, что оптимальное значение $k=0.2$. Меньшие и большие значения $k$ всегда выдают неоптимальные критерии качества.

Однако, оптимальное значение для $T$ получить не удалось, потому что все критерии качества строго улучшались с увеличением параметра $T$. Таким образом, чем больше значение $T$, тем качественнее система. Докажем это, сравнив критерии качества при $k=0.2, T=10$ и $k=0.2, T=1000$.

\subsubsection{Критерии качества при k=0.2 и T=10}

Cтатическая ошибка:

\begin{center}
$e=0$
\end{center}

Оценка быстродействия:

\begin{center}
$\Omega=\sqrt[3]{5}$
\end{center}

Корни характеристического уравнения:

\begin{equation*}
\begin{cases}
	\text{$p_1=-33.481218271995985$} \\
	\text{$p_2=-1.413101065200161$} \\
	\text{$p_3=-0.105680662803830$}
\end{cases}
\end{equation*}

Степень устойчивости:

\begin{center}
$min(|Re(p_1)|, |Re(p_2)|, |Re(p_3)|) = 0.105680662803830$
\end{center}

Колебательность системы:

\begin{equation*}
\begin{cases}
	\text{$Im(p_1)=0$} \\
	\text{$Im(p_2)=0$} \\
	\text{$Im(p_3)=0$}
\end{cases}
\end{equation*}

Показатель колебательности:

\begin{center}
$\theta=1$
\end{center}

\clearpage

Запас устойчивости по амплитуде:

\begin{center}
$C(\theta)=\infty$
\end{center}

Запас устойчивости по фазе:

\begin{center}
$\mu(\theta)=\frac{\pi}{3}$
\end{center}

Полоса пропускания:

\begin{center}
$0\leq\omega\leq0.0706403255$
\end{center}

Квадратичный критерий качества:

\begin{center}
$I=\int_{0}^{\infty}(h(t)-1(t))^2dt=0.1898876404$
\end{center}

Диаграмма боде:

\begin{figure}[h!]
	\centering
	\includegraphics[scale = 0.76]{images/bode10.png}
	\caption{Диаграмма боде для k=0.2 и T=10}
	\label{image:2}
\end{figure}

Шум:

\begin{figure}[h!]
	\centering
	\includegraphics[scale = 0.75]{images/noice.png}
	\caption{Шум, накладываемый на переходную характеристику}
	\label{image:3}
\end{figure}

\clearpage

Переходная характеристика без наложения шума:

\begin{figure}[h!]
	\centering
	\includegraphics[scale = 0.70]{images/step10.png}
	\caption{Переходная характеристика без наложения шума для k=0.2 и T=10}
	\label{image:4}
\end{figure}

Переходная характеристика с наложением шума:

\begin{figure}[h!]
	\centering
	\includegraphics[scale = 0.70]{images/stepnoice10.png}
	\caption{Переходная характеристика с наложением шума для k=0.2 и T=10}
	\label{image:5}
\end{figure}

\subsubsection{Критерии качества при k=0.2 и T=1000}

Cтатическая ошибка:

\begin{center}
$e=0$
\end{center}

Оценка быстродействия:

\begin{center}
$\Omega=\sqrt[3]{5}$
\end{center}

Корни характеристического уравнения:

\begin{equation*}
\begin{cases}
	\text{$p_1=-1020.097532402880$} \\
	\text{$p_2=-4.901467592120$} \\
	\text{$p_3=-0.01000005001$}
\end{cases}
\end{equation*}

Степень устойчивости:

\begin{center}
$min(|Re(p_1)|, |Re(p_2)|, |Re(p_3)|) = 0.01000005001$
\end{center}

Колебательность системы:

\begin{equation*}
\begin{cases}
	\text{$Im(p_1)=0$} \\
	\text{$Im(p_2)=0$} \\
	\text{$Im(p_3)=0$}
\end{cases}
\end{equation*}

Показатель колебательности:

\begin{center}
$\theta=1$
\end{center}

Запас устойчивости по амплитуде:

\begin{center}
$C(\theta)=\infty$
\end{center}

Запас устойчивости по фазе:

\begin{center}
$\mu(\theta)=\frac{\pi}{3}$
\end{center}

Полоса пропускания:

\begin{center}
$0\leq\omega\leq979.4641142403$
\end{center}

Квадратичный критерий качества:

\begin{center}
$I=\int_{0}^{\infty}(h(t)-1(t))^2dt=0.0005487688$
\end{center}

Диаграмма боде:

\begin{figure}[h!]
	\centering
	\includegraphics[scale = 0.56]{images/bode1000.png}
	\caption{Диаграмма боде для k=0.2 и T=1000}
	\label{image:6}
\end{figure}

Шум:

\begin{figure}[h!]
	\centering
	\includegraphics[scale = 0.56]{images/noice.png}
	\caption{Шум, накладываемый на переходную характеристику}
	\label{image:7}
\end{figure}

\clearpage

Переходная характеристика без наложения шума:

\begin{figure}[h!]
	\centering
	\includegraphics[scale = 0.70]{images/step1000.png}
	\caption{Переходная характеристика без наложения шума для k=0.2 и T=1000}
	\label{image:8}
\end{figure}

Переходная характеристика с наложением шума:

\begin{figure}[h!]
	\centering
	\includegraphics[scale = 0.70]{images/stepnoice1000.png}
	\caption{Переходная характеристика с наложением шума для k=0.2 и T=1000}
	\label{image:9}
\end{figure}


\section{Вывод}

При использовании изодромного звена в качестве управляющего устройства, оптимальные параметры не удалось установить однозначным образом. Как оказалось, параметр $T$ улучшает качественнные характеристики системы, поэтому при конструировании управляющего устройства следует выбирать максимально возможное $T$. Из эксперимента можно заметить, что при больших значениях $T$ улучшается степень устойчивости, увеличивается полоса пропускания, уменьшается воздействие шума, а также увеличивается скорость установления переходной характеристики.

Однако, параметр $k$ изодромного звена был получен однозначно: $k=0.2$. Любые отклонения от этого значения ухудшают качественные характеристики системы и вносят элемент колебательности.

Стоит отметить, что описанные правила для выбора $k$ и $T$ справедливы для только ОУ с конкретной переходной характеристикой, в то время как для других ОУ эти значения должны рассчитываться отдельно.




\end{document}