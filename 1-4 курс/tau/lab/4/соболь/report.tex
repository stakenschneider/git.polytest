\documentclass[12pt,a4paper]{article}
\usepackage[left=3cm, right=1cm, top=2cm, bottom=2cm]{geometry}
\usepackage[utf8]{inputenc}
\usepackage[russian]{babel}
\usepackage[OT1]{fontenc}
\usepackage{amsmath}
\usepackage{amsfonts}
\usepackage{amssymb}
\usepackage{csquotes}
\usepackage{graphicx}
\usepackage{float}
\usepackage{listings}
\graphicspath{{image/}}
\bibliographystyle{plain}

\usepackage{placeins}

\lstdefinestyle{base_listing}{
  extendedchars     = {true},
  inputencoding     = {utf8}, 
  basicstyle        = {\ttfamily \scriptsize},
  keywordstyle      = {\rmfamily \bfseries},
  commentstyle      = {\rmfamily \itshape},
  tabsize           = {2},
  flexiblecolumns   = {false},
  frame             = {single},
  showstringspaces  = {false},
  breaklines        = {true}, 
  breakatwhitespace = {true}
}

\lstdefinelanguage{LLVM-asm}
{
  morekeywords = {
    load, store, malloc, alloca, free, getelementptr,
    add, sub, insertvalue, extractvalue, icmp, call,
    define, void, global
  },
  sensitive   = false,
  morecomment = [l]{;}
}

\lstdefinestyle{crs_llvm}{
  style    = {base_listing},
  language = {LLVM-asm}
}

\lstdefinestyle{crs_cpp}{
  style    = {base_listing},
  language = {C++}
}

\lstdefinestyle{crs_bash}{
  style    = {base_listing},
  language = {bash}
}


\begin{document}

	\begin{titlepage}
		\begin{center}
			Санкт-Петербургский политехнический университет Петра Великого\\
			Институт компьютерных наук и технологий\\
			Кафедра компьютерных систем и программных технологий\\
			\vspace{6cm}
			Отчет по лабораторной работе №3\\
			\vspace{0.2cm}	
			по дисциплине "Основы теории управления"\\
			\vspace{0.5cm}	
			\Large
			Свойства объекта. Переход между формами ВСВ
			\small
		\end{center}
		\vspace{5cm}
		\begin{flushright}
			Выполнил: студент группы 43501/4	Соболь В.\\
			\vspace{0.5cm}			
			Преподаватель: Нестеров С.А.\\
		\end{flushright}
		\vspace{7cm}
		\begin{center}
			Санкт-Петербург\\
			2018 г.\\
		\end{center}
	\end{titlepage}
	\newpage

\def\contentsname{Содержание}

% Contents
\tableofcontents
\clearpage

\section{Цель работы}

Научиться определять оптимальные критерии качества для замкнутой системы.

\section{Программа работы}

\begin{itemize}
	\item Определить область устойчивости
	\item Определить величину статической ошибки.
	\item Получить корневые критерии качества.
	\item Получить частотные критерии качества.
	\item Получить интегральные критерии качества.
	\item Промоделировать процессы в системе при оптимальных параметрах при наличии шума и без.
\end{itemize}

\section{Индивидуальное задание}
Вид управляющего устройства: ПД (изодромное звено).\\
$x'' + 2x' + 0.75x = 0.75u\\
W(p) = \frac{0.75}{p^2 + 2p + 0.75}$

\newpage
\section{Ход работы}

\subsection{Исходные данные замкнутой системы}

Структура исследуемой системы с добавлением изодромного звена и шума:

\begin{figure}[h!]
	\centering
	\includegraphics[scale = 0.55]{images/w.png}
	\caption{Структурная схема системы}
	\label{image:1}
\end{figure}

\FloatBarrier
Определим передаточную функцию разомкнутой системы:

$W_p=\frac{k_pp+k_i}{p}\frac{0.75}{p^2 + 2p + 0.75}  =\frac{0.75(k_pp+k_i)}{p(p^2 + 2p + 0.75)}$

Определим характеристический полином замкнутой системы:

$D(p)=B(p)+C(p)=p(p^2 + 2p + 0.75)+0.75(k_pp+k_i)=0.75k_i + 0.75(1 + k_p)p + 2p^2 + p^3$

Определим передаточную функцию замкнутой системы:

$W_3=\frac{W_p}{1 + W_p}=\frac{B(p)}{B(p)+C(p)}=\frac{B(p)}{D(p)}=\frac{0.75(k_pp+k_i)}{Tp^3+(2T+1)p^2+(0.75T+0.75k+2)p+0.75} 
$
%$= \frac{3kp}{4Tp^3+4(2T+1)p^2+(3T+3k+8)p+3}$



\subsection{Статическая ошибка}

Для данной системы статическая ошибка вычисляется следующим образом:


$e=lim_{t\rightarrow\infty}\frac{U(t)}{1+W_p(t)}$

\noindentТак как система является астатической первого порядка, то $e \rightarrow 0$

\subsection{Корневые критерии качества}

Данная группа критериев применяется для оценки качества системы по корням характеристического полинома:



$D(p)=0.75k_i + 0.75(1 + k_p)p + 2p^2 + p^3$


\textbf{Оценка быстродействия} может производиться на основе величины:


$\Omega=\sqrt[n]{|p_1\cdot...\cdot p_n|}$


Для данной системы существует три корня:

$\Omega=\sqrt[3]{|p_1\cdot p_2\cdot p_3|}=\sqrt[3]{-0.75k_i}$
Видно, что система достигнет наилучшего быстродействия при значении $k_i = 0$.

\textbf{Степень устойчивости} системы определяется как абсолютное значение действительной части корней, ближайших к мнимой оси корня (к нулю):

$realPart = min(|Re(p_1)|, |Re(p_2)|, |Re(p_3)|)$

Таким образом, для получения оптимальных параметров $k_i$ и $k_P$, значение $realPart$ нужно минимизировать. В результате минимизации получились значения $k_i = 0$ и $k_p = 100$. 
Значения корней при полученных значениях параметров:

$\begin{cases}
 p_1 = 0\\p_2 = -1 + 8j\\ p_3 = -1 -8j 
 \end{cases}
 $

\textbf{Колебательность системы} определяется мнимыми частями корней. Для нулевой колебательности все мнимые части корней должны быть равны нулю.
Таким образом, для получения оптимальных параметров $k_i$ и $k_p$, должно быть выполнено условие:

$\begin{cases}
Imagine(p_1)=0\\Imagine(p_2)=0\\Imagine(p_3)=0
\end{cases}
$

Так как условие не выполняется, можно сказать, что при полученных значениях параметров
в системе имеется колебательность.



\subsection{Анализ полученной системы}


Эксперементально было выяснено, что оптимальное значение  $k = 0.04$ и $T = 35$.
Cтатическая ошибка: $e = 0$ 
Корни характеристического уравнения:

$\begin{cases}
 p_1 = 0.0133\\p_2 =0.0133\\ p_3 =-0.0267
 \end{cases}
 $
 

\begin{figure}[h!]
	\centering
	\includegraphics[scale = 0.80]{images/noise.png}
	\caption{Шум накладываемый на переходную характеристику}
	\label{image:9}
\end{figure}

\FloatBarrier


\begin{figure}[h!]
	\centering
	\includegraphics[scale = 0.80]{images/wnoise.png}
	\caption{Переходная характеристика без наложения шума}
	\label{image:9}
\end{figure}
\FloatBarrier

\begin{figure}[h!]
	\centering
	\includegraphics[scale = 0.80]{images/1.png}
	\caption{Переходная характеристика с наложением шума}
	\label{image:9}
\end{figure}
\FloatBarrier


Видно, что в промежуток времени, когда значение выходного сигнала возрастает, наличие шума практически не сказывается на поведении системы. Наибольшее
влияние обнаруживается, когда после этого система устремляется к значению в установившемся режиме. Также на графике видна колебательность системы.

\section{Вывод}

Анализ зависимости характеристик качества от параметров системы показал, что для исследуемой системы установить оптимальные параметры однозначно. Любые отклонения, в большую или меньшую сторону ухудшают качественные характеристики ситсемы и вносят элемент колебательности. 

По значениям корней можно сделать вывод, что система находится на апериодической границе
устойчивости. Также можно сказать, что в системе присутствует колебательность. Это
подтверждается результатами на графике.


\end{document}
