\chapter{Настройка рабочей среды}



Установка Python.
\begin{lstlisting}
sudo apt-get install python3
\end{lstlisting}


Установка для CPU.
Установка TensorFlow:
 \begin{lstlisting}
 pip install --ignore-installed --upgrade tensorflow
\end{lstlisting}

Для проверки корректности установки TensorFlow, запустите python и выполните программу:
 \begin{lstlisting}
 import tensorflow as tf
 hello = tf.constant('Hello, TensorFlow!')
 sess = tf.Session()
 print(sess.run(hello))
\end{lstlisting}
 
В результате должно быть напечатано:
 \begin{lstlisting}
 b'Hello, TensorFlow!'
 \end{lstlisting}
 
Установка для GPU.

Замечание! TensorFlow поддерживает GPU компании NVIDIA с CUDA Compute Capability 3.5 и выше. Видеокарты AMD и других производителей для TensorFlow не подходят.

Установка TensorFlow:
 \begin{lstlisting}
 pip install --upgrade tensorflow-gpu
\end{lstlisting}

Проверка установки. Для проверки корректности установки TensorFlow, запустите python и выполните программу:
 \begin{lstlisting}
 import tensorflow as tf
 tf.test.gpu_device_name()
\end{lstlisting}

Установка Keras.

Замечание! Начиная с версии 1.4 TensorFlow включает Keras. Поэтому отдельно устанавливать Keras не обязательно:
pip install keras 


Установка Jupyter Notebook.


Разверните среду разработки Python 3, в которую вы хотите установить Jupyter Notebook (в данном руководстве среда условно называется "my _env").

 \begin{lstlisting}
cd ~/environments

. my_env/bin/activate
\end{lstlisting}


Затем нужно обновить pip:
 \begin{lstlisting}
pip install --upgrade pip
\end{lstlisting}
Чтобы установить Jupyter Notebook, запустите:
 \begin{lstlisting}
pip install jupyter
\end{lstlisting}

Для того что бы запустить jupyter notebook, достаточно ввести в терминале 
 \begin{lstlisting}
jupyter notebook
\end{lstlisting}

Также можно ввести в строку браузера адрес http://localhost:8080 (по умолчанию).