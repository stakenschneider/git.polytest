\conclusion

В результате выполнения бакалаврской работы было произведено исследование технологий глубокого обучения: рассмотрены существующие модели нейронных сетей и поставлены задачи для которых они применяются, а также рассмотрены различные программы глубокого обучения. Был произведен анализ лабораторных работ, выложенных в открытый доступ.

Среди множества программных инструментов для глубокого обучения были выбраны библиотеки TensorFlow и Keras. Они отвечают потребностям разрабатываемого курса, а именно подходит для решения поставленных задач. 

Для визуализации полученных результатов была выбрана графическая веб-оболочка Jupyter Notebook. Проектирование проводилось в графической веб-оболочке Jupyter Notebook. В ней же и были сохранены сами лабораторные работы.


Практическая значимость разработки учебно-методического комплекса лабораторных работ, заключается в возможности внедрения её в учебный процесс.


Процесс прохождения курса лабораторных работ по дисциплине "исследование моделей глубокого обучения" состоит из 3 работ. Каждая лабораторная направлена на изучение различных задач: классификации, локализации изображений и прогнозирование временных рядов. По их завершению обучающиеся научатся сводить разнообразные задачи к формальным постановкам задач машинного обучения и применять необходимые технологии и схемы для их решения.

Задачи бакалаврской работы были выполнены, а цели - достигнуты.

