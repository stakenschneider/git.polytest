\intro 

В настоящее время глубокое обучение является активно развивающейся областью научных исследований. Своими успехами глубокое обучение обязано постоянно увеличивающемуся объему обучающих данных и относительно дешевым графическим процессорам, позволяющим построить эффективную процедуру вычислений.

Глубокое обучение рассматривает методы моделирования высокоуровневых абстракций с помощью множества последовательных нелинейных трансформаций, которые, как правило, представляются в виде искусственных нейронных сетей. На сегодняшний день нейросети успешно используются для решения таких задач, как прогнозирование, распознавание образов и ряда других.


Глубокое обучение является подмножеством методов машинного обучения, в которых применяются искусственные нейронные сети, построенные на базе аналогии со структурой нейронов человеческого мозга. Термин «глубокий» подразумевает наличие большого числа слоев в нейронной сети. 

В компаниях Google, Microsoft, Amazon, Apple, Facebook и многих других методы глубокого обучения постоянно используются для анализа больших массивов данных. Теперь эти знания и навыки вышли за рамки чисто академических исследований и стали достоянием крупных промышленных компаний. 

Актуальность темы глубокого обучения подтверждается регулярным появлением статей на данную тему.

Целью дипломной работы  является разработка учебно-методические средства по исследованию моделей глубокого обучения. 
В соответствии с целью исследования были поставлены следующие задачи:
\begin{itemize}
\item проанализировать существующие лабораторные работы;
\item  сравнить и выбрать технологии моделирования глубоких  нейронных сетей;
\item разработать и протестировать несколько лабораторных работ.
\end{itemize}