\documentclass[14pt,a4paper,report]{report}
\usepackage[a4paper, mag=1000, left=2.5cm, right=1cm, top=2cm, bottom=2cm, headsep=0.7cm, footskip=1cm]{geometry}
\usepackage[utf8]{inputenc}
\usepackage[english,russian]{babel}
\usepackage{indentfirst}
\usepackage[dvipsnames]{xcolor}
\usepackage[colorlinks]{hyperref}
\usepackage{listings} 
\usepackage{fancyhdr}
\usepackage{caption}
\usepackage{amsmath}
\usepackage{latexsym}
\usepackage{graphicx}
\usepackage{amsmath}
\hypersetup{
	colorlinks = true,
	linkcolor  = black
}

\usepackage{titlesec}
\titleformat{\chapter}
{\Large\bfseries} % format
{}                % label
{0pt}             % sep
{\huge}           % before-code


\DeclareCaptionFont{white}{\color{white}} 

% Listing description
\usepackage{listings} 
\DeclareCaptionFormat{listing}{\colorbox{gray}{\parbox{\textwidth}{#1#2#3}}}
\captionsetup[lstlisting]{format=listing,labelfont=white,textfont=white}
\lstset{ 
	% Listing settings
	inputencoding = utf8,			
	extendedchars = \true, 
	keepspaces = true, 			  	 % Поддержка кириллицы и пробелов в комментариях
	language = Python,            	 	 % Язык программирования (для подсветки)
	basicstyle = \small\sffamily, 	 % Размер и начертание шрифта для подсветки кода
	numbers = left,               	 % Где поставить нумерацию строк (слева\справа)
	numberstyle = \tiny,          	 % Размер шрифта для номеров строк
	stepnumber = 1,               	 % Размер шага между двумя номерами строк
	numbersep = 5pt,              	 % Как далеко отстоят номера строк от подсвечиваемого кода
	backgroundcolor = \color{white}, % Цвет фона подсветки - используем \usepackage{color}
	showspaces = false,           	 % Показывать или нет пробелы специальными отступами
	showstringspaces = false,    	 % Показывать или нет пробелы в строках
	showtabs = false,           	 % Показывать или нет табуляцию в строках
	frame = single,              	 % Рисовать рамку вокруг кода
	tabsize = 2,                  	 % Размер табуляции по умолчанию равен 2 пробелам
	captionpos = t,             	 % Позиция заголовка вверху [t] или внизу [b] 
	breaklines = true,           	 % Автоматически переносить строки (да\нет)
	breakatwhitespace = false,   	 % Переносить строки только если есть пробел
	escapeinside = {\%*}{*)}      	 % Если нужно добавить комментарии в коде
}

\begin{document}

\def\contentsname{Содержание}

% Titlepage
\begin{titlepage}
	\begin{center}
		\textsc{Санкт-Петербургский Политехнический 
			Университет Петра Великого\\[5mm]
			Кафедра компьютерных систем и программных технологий}
		
		\vfill
		
		\textbf{Отчёт по лабораторной работе №2\\[3mm]
			Курс: «Базы данных»\\[3mm]
			Тема: «Создание интерактивного генератора данных»\\[35mm]
			}
	\end{center}
	
	\hfill
	\begin{minipage}{.5\textwidth}
		Выполнил студент:\\[2mm] 
		Бояркин Никита Сергеевич\\
		Группа: 43501/3\\[5mm]
		
		Проверил:\\[2mm] 
		Мяснов Александр Владимирович
	\end{minipage}
	\vfill
	\begin{center}
		Санкт-Петербург\\ \the\year\ г.
	\end{center}
\end{titlepage}

% Contents
\tableofcontents
\clearpage

\chapter{Лабораторная работа №2}

\section{Цель работы}

Получить практические навыки работы с БД путем создания собственного интерактивного генератора данных на языке программирования python.

\section{Программа работы}

\begin{itemize}
	\item Разработать интерактивный генератор данных.
	\item Заполнить таблицы данными с помощью генератора.
\end{itemize}

\section{Программное окружение}

\begin{itemize}
	\item Python 3.6
	\item Django 1.10.5
	\item Psycopg 2.6.2
	\item PostgreSQL 9.5.6
\end{itemize}

\section{Ход работы}

\subsection{Создание проекта, конфигурация сервера}

Генератор представляет собой management команду, которая принимает два аргумента: название таблицы для добавления и количество данных для генерации. 

Данные генерируются для всех таблиц пропорционально их собственному коэффициенту. Например, при генерации в таблице болезней $N$ строк, в таблице лекарств будет сгенерировано $2N$ строк, в таблице поставок будет сгенерировано $4N$ строк и так далее для каждой таблице. Такой метод генерации сохраняет логику базы данных, ведь количество поставок лекарств обычно больше количества лекарств, а количество лекарств обычно больше количества болезней.

Кроме пропорционального заполнения, генератор обеспечивает логическую целостность данных в отдельных таблицах. Это ознчачает, например, что таблицы показаний и противопоказаний не содержат одинаковых пар лекарство + болезнь; таблица несовместимости лекарств не содержит повторяющихся значений; цены, даты и количества лекарств содержатся в определенных границах и др.

Полный листинг команды генерации данных:

\lstinputlisting{listings/generate.py}

Пример выполнения команды:

\lstinputlisting{listings/result.log}

После этого были сгенерированы данные пропорционально весу каждой таблицы:

\begin{itemize}
	\item 2000 лекарств.
	\item 1000 болезней.
	\item 250 несовместимостей лекарств.
	\item 250 противопоказаний.
	\item 250 диагнозов (показаний).
	\item 250 аптек.
	\item 250 поставщиков.
	\item 4000 поставок.
	\item 250 клиентов.
	\item 500 частей заказов.
	\item 2000 заказов.
\end{itemize}	

\section{Вывод}

Написание собственного генератора намного более гибкое решение, чем добавление данных вручную. Это обусловлено тем, что при тестировании нас обычно не волнуют точные значения имен, цен и прочих параметров, в то время как пропорции данных между таблицами, контроль повторных значений и неявные зависимости между таблицами важны при тестировании.

\end{document}