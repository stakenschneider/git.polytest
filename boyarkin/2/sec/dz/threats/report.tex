\documentclass[14pt,a4paper,report]{report}
\usepackage[a4paper, mag=1000, left=2.5cm, right=1cm, top=2cm, bottom=2cm, headsep=0.7cm, footskip=1cm]{geometry}
\usepackage[utf8]{inputenc}
\usepackage[english,russian]{babel}
\usepackage{indentfirst}
\usepackage[dvipsnames]{xcolor}
\usepackage[colorlinks]{hyperref}
\usepackage{listings} 
\usepackage{fancyhdr}
\usepackage{caption}
\usepackage{graphicx}
\hypersetup{
	colorlinks = true,
	linkcolor  = black
}

\usepackage{titlesec}
\titleformat{\chapter}
{\Large\bfseries} % format
{}                % label
{0pt}             % sep
{\huge}           % before-code


\DeclareCaptionFont{white}{\color{white}} 

% Listing description
\usepackage{listings} 
\DeclareCaptionFormat{listing}{\colorbox{gray}{\parbox{\textwidth}{#1#2#3}}}
\captionsetup[lstlisting]{format=listing,labelfont=white,textfont=white}
\lstset{ 
	% Listing settings
	inputencoding = utf8,			
	extendedchars = \true, 
	keepspaces = true, 			  	 % Поддержка кириллицы и пробелов в комментариях
	language = C,            	 	 % Язык программирования (для подсветки)
	basicstyle = \small\sffamily, 	 % Размер и начертание шрифта для подсветки кода
	numbers = left,               	 % Где поставить нумерацию строк (слева\справа)
	numberstyle = \tiny,          	 % Размер шрифта для номеров строк
	stepnumber = 1,               	 % Размер шага между двумя номерами строк
	numbersep = 5pt,              	 % Как далеко отстоят номера строк от подсвечиваемого кода
	backgroundcolor = \color{white}, % Цвет фона подсветки - используем \usepackage{color}
	showspaces = false,           	 % Показывать или нет пробелы специальными отступами
	showstringspaces = false,    	 % Показывать или нет пробелы в строках
	showtabs = false,           	 % Показывать или нет табуляцию в строках
	frame = single,              	 % Рисовать рамку вокруг кода
	tabsize = 2,                  	 % Размер табуляции по умолчанию равен 2 пробелам
	captionpos = t,             	 % Позиция заголовка вверху [t] или внизу [b] 
	breaklines = true,           	 % Автоматически переносить строки (да\нет)
	breakatwhitespace = false,   	 % Переносить строки только если есть пробел
	escapeinside = {\%*}{*)}      	 % Если нужно добавить комментарии в коде
}

\begin{document}

\def\contentsname{Содержание}

% Titlepage
\begin{titlepage}
	\begin{center}
		\textsc{Санкт-Петербургский Политехнический 
			Университет Петра Великого\\[5mm]
			Кафедра компьютерных систем и программных технологий}
		
		\vfill
		
		\textbf{Домашнее задание №1\\[3mm]
			Курс: «Защита информации»\\[6mm]
			Тема: «Угрозы информационной безопасности»\\[35mm]
		}
	\end{center}
	
	\hfill
	\begin{minipage}{.5\textwidth}
		Выполнил студент:\\[2mm] 
		Бояркин Никита Сергеевич\\
		Группа: 43501/3\\[5mm]
		
		Проверил:\\[2mm] 
		Новопашенный Андрей Гелиевич
	\end{minipage}
	\vfill
	\begin{center}
		Санкт-Петербург\\ \the\year\ г.
	\end{center}
\end{titlepage}

\chapter{Домашнее задание №1}

\section{Угрозы для корпоративных сетей}

\begin{itemize}
	\item \emph{Физическая угроза данным} - Риск потерять данные после сбоя физического носителя (происходит обычно при авариях, пожарах, потопах и др.).
	\item \emph{Угроза со стороны персонала} - Вызвана пренебрежением правилами, личной выгодой или саботажем.
	\item \emph{Вредоносное ПО и хакерские атаки} - Попытки хищения или нарушения целостности данных программными методами.
\end{itemize}

\subsection{Защита корпоративных сетей}

\begin{itemize}
	\item \emph{Защита от физической угрозы данным} - Резервное копирование данных (на носители в различных местах), учет и уход за носителем и сервисным оборудованием (системы кондиционирования, отопления, освещения и др.).
	\item \emph{Защита от персонала} - Контроль, санкции за пренебрежение мерами безопасности, NDA.
	\item \emph{Защита от вредоносного ПО и хакерских атак} - Экранирование, обновляемый антивирус, наличие системного администратора.
\end{itemize}

\section{Угрозы для частных лиц}

\begin{itemize}
	\item \emph{Беспечность пользователя} - Публикация информации о кредитных картах, билетах, паролях и др. в социальных сетях и других информационных ресурсах.
	\item \emph{Вредоносное ПО} - Обычно, загрузка вируса из сети.
	\item \emph{Физическая угроза данным} - Риск потерять данные после сбоя физического носителя (чаще всего выход из строя SSD, потеря флешки или ноутбука).
\end{itemize}

\subsection{Защита частных лиц}

\begin{itemize}
	\item \emph{Защита от беспечности пользователя} - не быть беспечным.
	\item \emph{Защита от вредоносного ПО} - Наличие обновляемого антивируса, современного браузера, блокировщика рекламы.
	\item \emph{Защита от физической угрозы данным} - Резервное копирование данных (на носители в различных местах), уход за носителем. Оставлять контактную информацию на флешках и ноутбуках в случае потери.
\end{itemize}

\end{document}