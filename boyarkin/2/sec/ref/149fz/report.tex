\documentclass[14pt,a4paper,report]{article}
\usepackage[a4paper, mag=1000, left=2.5cm, right=1cm, top=2cm, bottom=2cm, headsep=0.7cm, footskip=1cm]{geometry}
\usepackage[utf8]{inputenc}
\usepackage[english,russian]{babel}
\usepackage{indentfirst}
\usepackage[dvipsnames]{xcolor}
\usepackage[colorlinks]{hyperref}
\usepackage{listings} 
\usepackage{fancyhdr}
\usepackage{caption}
\usepackage{graphicx}
\usepackage{csquotes}
\hypersetup{
	colorlinks = true,
	linkcolor  = black
}

\usepackage{titlesec}
\titleformat{\chapter}
{\Large\bfseries} % format
{}                % label
{0pt}             % sep
{\huge}           % before-code


\DeclareCaptionFont{white}{\color{white}} 

% Listing description
\usepackage{listings} 
\DeclareCaptionFormat{listing}{\colorbox{gray}{\parbox{\textwidth}{#1#2#3}}}
\captionsetup[lstlisting]{format=listing,labelfont=white,textfont=white}
\lstset{ 
	% Listing settings
	inputencoding = utf8,			
	extendedchars = \true, 
	keepspaces = true, 			  	 % Поддержка кириллицы и пробелов в комментариях
	language = C,            	 	 % Язык программирования (для подсветки)
	basicstyle = \small\sffamily, 	 % Размер и начертание шрифта для подсветки кода
	numbers = left,               	 % Где поставить нумерацию строк (слева\справа)
	numberstyle = \tiny,          	 % Размер шрифта для номеров строк
	stepnumber = 1,               	 % Размер шага между двумя номерами строк
	numbersep = 5pt,              	 % Как далеко отстоят номера строк от подсвечиваемого кода
	backgroundcolor = \color{white}, % Цвет фона подсветки - используем \usepackage{color}
	showspaces = false,           	 % Показывать или нет пробелы специальными отступами
	showstringspaces = false,    	 % Показывать или нет пробелы в строках
	showtabs = false,           	 % Показывать или нет табуляцию в строках
	frame = single,              	 % Рисовать рамку вокруг кода
	tabsize = 2,                  	 % Размер табуляции по умолчанию равен 2 пробелам
	captionpos = t,             	 % Позиция заголовка вверху [t] или внизу [b] 
	breaklines = true,           	 % Автоматически переносить строки (да\нет)
	breakatwhitespace = false,   	 % Переносить строки только если есть пробел
	escapeinside = {\%*}{*)}      	 % Если нужно добавить комментарии в коде
}

\begin{document}

\def\contentsname{Содержание}

% Titlepage
\begin{titlepage}
	\begin{center}
		\textsc{Санкт-Петербургский Политехнический 
			Университет Петра Великого\\[5mm]
			Кафедра компьютерных систем и программных технологий}
		
		\vfill
		
		\textbf{Реферат\\[3mm]
			Курс: «Защита информации»\\[6mm]
			Тема: «Редакции федерального закона о защите информации с 2012 года»\\[35mm]
		}
	\end{center}
	
	\hfill
	\begin{minipage}{.4\textwidth}
		Выполнил студент:\\[2mm] 
		Бояркин Никита Сергеевич\\
		Группа: 43501/3\\[5mm]
		
		Проверил:\\[2mm] 
		Новопашенный Андрей Гелиевич
	\end{minipage}
	\vfill
	\begin{center}
		Санкт-Петербург\\ \the\year\ г.
	\end{center}
\end{titlepage}

% Contents
\tableofcontents
\clearpage


\section{Введение}

В данном реферате обозреваются изменения федерального закона 149 «Об информации, информационных технологиях и о защите информации», начиная с 2012 года. Сам закон был принят Государственной Думой 8 июля 2006 года и одобрен Советом Федерации 14 июля 2006 года. Всего в данный закон было внесено 20 изменений [1]. После 2012 года было внесено 17 изменений с 27 июля 2012 года по 19 декабря 2016 года [1].


\section{Изменения в 139-ФЗ «О защите детей от информации, причиняющей вред их здоровью и развитию» от 28.07.2012}

\subsection{Изменения в статье 2 «Основные понятия, используемые в настоящем Федеральном законе»}

В редакции этого закона были добавлены некоторые основные понятия, такие как:

\begin{enumerate}
	\item Сайт в сети Интернет;
	\item Страница сайта в сети Интернет;
	\item Доменное имя;
	\item Сетевой адрес;
	\item Владелец сайта в сети Интернет;
	\item Провайдер хостинга.
\end{enumerate}

Формулировки некоторых понятий не совсем полные, например формулировка доменного имени:

\begin{displayquote}
\emph{Доменное имя - обозначение символами, предназначенное для адресации сайтов в сети "Интернет" в целях обеспечения доступа к информации, размещенной в сети "Интернет". [2]}
\end{displayquote}

В этой формулировке нет никакой определяющей технической информации (например о DNS), поэтому под определение "доменного имени" попадают любые символьные комбинации, которые указывают на сайт в сети "Интернет". 

\subsection{Статья 15.1 «Единый реестр доменных имен»}

Была добавлена статья 15.1, которая определяет реестр доменных имен, указателей страниц сайтов в сети "Интернет" и сетевых адресов, позволяющих идентифицировать сайты в сети "Интернет", содержащие информацию, распространение которой в Российской Федерации запрещено [2]. Статья определяет следующие операции:

\begin{itemize}
	\item Основания для включения в реестр;
	\item Процедура включения в реестр;
	\item Процедура исключения из реестра;
	\item Кто контролирует над реестр;
\end{itemize}

\section{50-ФЗ «О внесении изменений в отдельные законодательные акты Российской Федерации в части ограничения распространения информации о несовершеннолетних» от 05.04.2013}

\subsection{Изменения в статье 15.1 «Основные понятия, используемые в настоящем Федеральном законе»}

В данной редакции было добавлено еще одно основание для включения в реестр:

\begin{displayquote}
\emph{Информация о несовершеннолетнем, пострадавшем в результате противоправных действий (бездействия), распространение которой запрещено федеральными законами. [3]}
\end{displayquote}

\section{112-ФЗ «О внесении изменений в Федеральный закон "Об информации, информационных технологиях и о защите информации"» от 07.06.2013}

\subsection{Изменения в статье 2 «Основные понятия, используемые в настоящем Федеральном законе»}

В данном законе было изменено определение понятия «Сайт в сети "Интернет"», формулировка "через сеть "Интернет" были изменена на "посредством информационно-телекоммуникационной сети "Интернет" [4].

Также было добавлено еще одно основное понятие:

\begin{displayquote}
	\emph{Единая система идентификации и аутентификации - федеральная государственная информационная система, порядок использования которой устанавливается Правительством Российской Федерации и которая обеспечивает в случаях, предусмотренных законодательством Российской Федерации, санкционированный доступ к информации, содержащейся в информационных системах. [4]}
\end{displayquote}

\subsection{Изменения в статье 7 «Общедоступная информация»}

Было добавлено три части:

\begin{itemize}
	\item Часть 4 статьи 7 определяет, что информация, которая размещается на сайтах в сети "Интернет" является общедоступной, размещаемой в форме открытых данных [4].
	\item Часть 5 статьи 7 определяет, что если информация, которая размещается на сайтах в сети "Интернет" является государственной тайной, то размещение этих данных должно быть прекращено [4].
	\item Часть 6 статьи 7 определяет, что если информация, которая размещается на сайтах в сети "Интернет" нарушает права обладателей информации, то размещение этих данных должно быть прекращено по решению суда [4].
\end{itemize}

\subsection{Изменения в статье 14 «Государственные информационные системы»}

Согласно новым дополнениям, общедоступная информация должна размещаться с учетом требований законодательства России о государственной тайне. Также общедоступная информация не должна нарушать прав обладателей информации ограниченной федеральными законами [4].

\section{187-ФЗ «О внесении изменений в Федеральный закон "О внесении изменений в отдельные законодательные акты Российской Федерации по вопросам защиты интеллектуальных прав в информационно-телекоммуникационных сетях» от 02.07.2013}

\subsection{Статья 15.2 «Ограничение доступа к информации, распространяемой с нарушением исключительных прав на фильмы»}

В статье описывается:

\begin{enumerate}
	\item Права правообладателя;
	\item Действия, которые необходимо предпринять правообладателю для ограничения доступа;
	\item Действия провайдера хостинга по ограничению доступа;
	\item Действия владельца ресурса по удалению общедоступной информации.
\end{enumerate}

Важно отметить, что под этот закон попадают все сайты в сети "Интернет". Если на сайте замечена запрещенная информация, то владельцу ресурса отравляется электронное письмо, с просьбой удалить запрещенную информацию. Если в течении определенного срока эта информация не удалена, то провайдеры обязаны блокировать этот сайт. Закон не подразумевает блокирование отдельных страниц на сайте [5].

\subsection{Изменения в статье 17 «Ограничение доступа к информации, распространяемой с нарушением исключительных прав на фильмы»}

Была добавлена часть 4:

\begin{displayquote}
	\emph{Провайдер хостинга и владелец сайта в сети "Интернет" не несут ответственность перед правообладателем и перед пользователем за ограничение доступа к информации и (или) ограничение ее распространения в соответствии с требованиями настоящего Федерального закона. [5]}
\end{displayquote}

\section{396-ФЗ «О внесении изменений в отдельные законодательные акты Российской Федерации» от 28.12.2013}

\subsection{Изменения в статье 14 «Государственные информационные системы»}

Была добавлена часть 2:

\begin{displayquote}
	\emph{Государственные информационные системы создаются и эксплуатируются с учетом требований, предусмотренных законодательством Российской Федерации о контрактной системе в сфере закупок товаров, работ, услуг для обеспечения государственных и муниципальных нужд. [6]}
\end{displayquote}

\section{398-ФЗ «О внесении изменений в Федеральный закон "Об информации, информационных технологиях и о защите информации"» от 28.12.2013}

\subsection{Статья 15.3 «Порядок ограничения доступа к информации, распространяемой с нарушением закона»}

Под ограничение доступа к информации попадают призывы к:

\begin{itemize}
	\item Осуществлению экстремистской деятельности;
	\item Массовым беспорядкам;
	\item Участию в публичных мероприятиях, проводимых с нарушением установленного порядка.
\end{itemize}

Доступ к ресурсу, содержащему данную информации осуществляется незамедлительно, без контакта с владельцем ресурса, однако доступ к ресурсу может быть восстановлен через федеральные органы исполнительной власти [7].

\subsection{Изменения в статье 15.1 «Единый реестр доменных имен»}

Добавлена часть, в которой указывается, что статья 15.1 не применяется к информации, порядок ограничения доступа к которой предусмотрен статьей 15.3 [7].

\section{97-ФЗ «О внесении изменений в Федеральный закон "Об информации, информационных технологиях и о защите информации"» от 05.05.2014}

\subsection{Статья 10.1 «Обязанности организатора распространения информации в сети "Интернет"»}

Была добавлена статья 10.1, которая определяет понятие организатора распространения информации в сети "Интернет":

\begin{displayquote}
	\emph{Организатором распространения информации в сети "Интернет" является лицо, осуществляющее деятельность по обеспечению функционирования информационных систем и (или) программ для электронных вычислительных машин, которые предназначены и (или) используются для приема, передачи, доставки и (или) обработки электронных сообщений пользователей сети "Интернет". [8]}
\end{displayquote}

Обязанности организатора распространения информации:

\begin{itemize}
	\item Уведомление федеральных органов о начале осуществления деятельности по организации распространения информации;
	\item Хранение информации пользователей за последние полгода, а также предоставления ее правоохранительным органам;
	\item Реализация требований к оборудованию и программно-техническим средствам.
\end{itemize}

Данные требования не распространяются на операторов государственных информационных систем, операторов муниципальных информационных систем, операторов связи, оказывающих услуги связи на основании соответствующей лицензии [8].

\subsection{Статья 10.2 «Распространение блогером общедоступной информации»}

Была добавлена статья 10.2, которая определяет понятие блогера общедоступной информации:

\begin{displayquote}
	\emph{Владелец сайта и (или) страницы сайта в сети "Интернет", на которых размещается общедоступная информация и доступ к которым в течение суток составляет более трех тысяч пользователей сети "Интернет" (далее - блогер). [8]}
\end{displayquote}

Также статья определяет обязанности блогера [8]:

\begin{itemize}
	\item Не допускать использование сайта или страницы сайта в сети "Интернет" в целях совершения уголовно наказуемых деяний.
	\item Соблюдать запреты, ограничения и требования предусмотренные законодательством Российской Федерации о референдуме, выборах, порядка распространения массовой информации.
	\item Соблюдать права и законные интересы граждан и организаций.
	\item Проверять достоверность размещаемой общедоступной информации.
	\item Не допускать распространение информации о частной жизни гражданина.
	\item Соблюдать запреты, ограничения и требования предусмотренные законодательством Российской Федерации о референдуме, выборах, порядка распространения массовой информации.
\end{itemize}

\section{222-ФЗ «О внесении изменений в Федеральный закон "О государственном регулировании деятельности по организации и проведению азартных игр"» от 21.07.2014}

\subsection{Изменения в статье 15.1 «Основные понятия, используемые в настоящем Федеральном законе»}

В данной редакции еще одним основанием для добавления в реестр являются азартные игры и лотереи (244-ФЗ и 138-ФЗ) в сети "Интернет" [9].

\section{242-ФЗ «О внесении изменений в отдельные законодательные акты Российской Федерации в части уточнения порядка обработки персональных данных в информационно-телекоммуникационных сетях"» от 21.07.2014}

\subsection{Статья 15.5 «Порядок ограничения доступа к информации, обрабатываемой с нарушением законодательства РФ в области персональных данных»}

В новой статье определяется "Реестр нарушителей прав субъектов персональных данных". Аналогично со статьей 15.1 определяются следующие операции [10]:

\begin{itemize}
	\item Основания для включения в реестр;
	\item Процедура включения в реестр;
	\item Процедура исключения из реестра;
	\item Кто контролирует над реестр;
\end{itemize}

\section{364-ФЗ «О внесении изменений в Федеральный закон "Об информации, информационных технологиях и о защите информации» от 24.11.2014}

\subsection{Изменения в статье 15.2 «Порядок ограничения доступа к информации, распространяемой с нарушением авторских и (или) смежных прав»}

В новой редакции "исключительные" права заменяются на "авторские и (или) смежные". Также было расширено действие авторских и (или) смежных прав. Если ранее права распространялись на фильмы, кинофильмы, телефильмы, то сейчас это распространяется на любой объект, на который есть права, за исключением фотографических произведений [11].

\subsection{Статья 15.6 «Порядок ограничения доступа к сайтам в сети "Интернет", на которых неоднократно и неправомерно размещалась информация, содержащая объекты авторских и (или) смежных прав»}

Была добавлена статья 15.6, которая трактует порядок постоянного ограничения доступа к сайту неоднократно нарушавшего авторские и (или) смежные права. Снятие ограничений к таким сайтам не допускается. Список таких сайтов размещается на официальном сайте федерального органа исполнительной власти [11].

\subsection{Статья 15.7 «Внесудебные меры по прекращению нарушения авторских и (или) смежных прав в информационно-телекоммуникационных сетях»}

Была добавлена статья 15.7. Согласно данной статье, правообладатель может потребовать от владельца сайта, удаления информации на которую у него имеются права [11].

\section{188-ФЗ «О внесении изменений в Федеральный закон "Об информации, информационных технологиях и о защите информации"» от 29.06.2015}

\subsection{Статья 12.1 «Особенности государственного регулирования в сфере использования российских программ для электронных вычислительных машин и баз данных»}

В данной редакции определяется "Реестр российского программного обеспечения" [13]. Аналогично со статьей 15.1 определяются следующие операции:

\begin{itemize}
	\item Основания для включения в реестр;
	\item Процедура включения в реестр;
	\item Процедура исключения из реестра;
	\item Кто контролирует над реестр;
\end{itemize}

Для включения в реестр, программный продукт должен обладать следующими требованиями [13]:

\begin{itemize}
	\item Исключительное право на программу, должно принадлежать одному или нескольким из следующих лиц:
	
	\begin{itemize}
		\item Российской Федерации, субъекту Российской Федерации, муниципальному образованию
		\item Российской некоммерческой организации, решения которой иностранное лицо не имеет возможности определять
		\item Российской коммерческой организации, в которой суммарная доля Российских организаций более 50%
		\item Гражданину России
	\end{itemize}
	
	\item Программа уже введена в гражданский оборот
	\item Менее 30\% выручки уходит за рубеж
	\item Сведения о программе не содержат государственной тайны
\end{itemize}

Решения об отказе во включении в реестр, может быть обжаловано правообладателем в суде [13].

\section{263-ФЗ «О внесении изменений в отдельные законодательные акты Российской Федерации в части отмены ограничений на использование электронных документов при взаимодействии физических и юридических лиц с органами государственной власти и органами местного самоуправления"» от 13.07.2015}

\subsection{Статья 11.1 «Обмен информацией в форме электронных документов при осуществлении полномочий органов государственной власти и органов местного самоуправления»}

Согласно данной статье, органы власти или местного самоуправления обязаны предоставить по выбору граждан и организаций информацию в форме электронных документов, подписанных усиленной квалифицированной электронной подписью, и (или) документов на бумажном носителе [14].

\section{264-ФЗ «О внесении изменений в Федеральный закон "Об информации, информационных технологиях и о защите информации"» от 13.07.2015}

\subsection{Изменения в статье 2 «Основные понятия, используемые в настоящем Федеральном законе»}

Было добавлено определение поисковой системы:

\begin{displayquote}
	\emph{Поисковая система - информационная система, осуществляющая по запросу пользователя поиск в сети "Интернет" информации определенного содержания и предоставляющая пользователю сведения об указателе страницы сайта в сети "Интернет" для доступа к запрашиваемой информации, расположенной на сайтах в сети "Интернет", принадлежащих иным лицам, за исключением информационных систем, используемых для осуществления государственных и муниципальных функций, оказания государственных и муниципальных услуг, а также для осуществления иных публичных полномочий, установленных федеральными законами. [15]}
\end{displayquote}

\subsection{Статья 10.3 «Обязанности оператора поисковой системы»}

В данной статье описываются обязанности оператора поисковой системы. Каждый гражданин может обратиться с требованиями прекратить выдачу ссылок на страницу сайта с информацией о заявителе, которая нарушает законодательство. Заявитель вправе обратиться в суд, в случае отказа оператора [15].

\section{208-ФЗ «О внесении изменений в Федеральный закон "Об информации, информационных технологиях и о защите информации"» от 23.06.2016}

\subsection{Статья 10.4 «Особенности распространения информации новостным агрегатором»}

Была добавлена статья 10.4, в которой введено понятие новостного агрегатора [16]:

\begin{displayquote}
	\emph{Новостным агрегатором является владелец сайта или страницы в сети "Интернет которая используется для распространения новостной информации на русском языке, доступ к которой в течении суток составляет более одного миллиона пользователей.}
\end{displayquote}

Новостной агрегатор обязан [16]:

\begin{enumerate}
	\item Не распространять государственную тайну, не призывать к террористической деятельности или оправдывать терроризм
	\item Не пропагандировать порнографию, культ насилия и жестокости, и материалов, содержащих нецензурную брань
	\item Проверять достоверность распространяемых сведений до их распространения
	\item Не фальсифицировать сведения и не распространять недостоверную информацию под видом достоверной
	\item Не допускать опорочивание гражданина или группы граждан по религии, полу, расе и т.д.
	\item Соблюдать запреты, ограничения и требования законодательства России
	\item Обеспечить возможность для обратной связи
	\item Хранить в течение шести месяцев распространенную ими новостную информацию, сведения об источнике ее получения, сведения о сроках ее распространения, а также обеспечить доступ федеральных органов к этой информации
\end{enumerate}

Существует соответствующий реестр новостных агрегаторов, необходимый для контроля, надзора и мониторинга этих информационных ресурсов. В данной статье описан процесс признания информационного ресурса - новостным агрегатором, и последующего внесения в реестр. А также описан процесс исключения из реестра и снятие статуса новостного агрегатора [16].

\section{374-ФЗ «О внесении изменений в Федеральный закон "О противодействии терроризму"» от 06.07.2016}

\subsection{Изменения в статье 10.1 «Обязанности организатора распространения информации в сети "Интернет"»}

В новой редакции организаторам распространения информации необходимо хранить не только пересылаемую пользователями информацию в течение полугода, но и хранить факты подобных действий в течении одного года, а также предоставлять ее государственным органам, осуществляющим оперативно-разыскную деятельность
или обеспечение безопасности Российской Федерации, в случаях, установленных федеральными законами [17].

Также добавлена обязанность кодировать сохраняемые данные пользователей, а при предоставлении этих данных федеральным органам, обеспечивать возможность декодирования этой информации.

\section{442-ФЗ «О внесении изменения в статью 15.1 Федерального закона "Об информации, информационных технологиях и о защите информации"» от 19.12.2016}	

\subsection{Изменения в статье 15.1 «Единый реестр доменных имен»}

В данной редакции была исправлена формулировка основания для включения в реестр (статья 15.1, часть 5, подпункт "б"):

\begin{displayquote}
	\emph{Информация о способах, методах разработки, изготовления и использования наркотических средств, психотропных веществ и их прекурсоров, новых потенциально опасных психоактивных веществ, местах их приобретения, способах и местах культивирования наркосодержащих растений. [18]}
\end{displayquote}
	
\section{Заключение}

Изменения, вносимые с 2012 года добавили множество проблем как провайдерам, так и владельцам информационных ресурсов:

\begin{itemize}
	\item Провайдеры обязаны хранить огромное количество информации о пользователях, а также блокировать доступ к информационным ресурсам для пользователей. 
	\item Владелец информационного ресурса должен постоянно удалять запрещенную в Российской Федерации информацию в течении небольшого промежутка времени после уведомления. Даже тот факт, что уведомление не дойдет до владельца сайта или то, что владелец может не знать законов Российской Федерации не остановит блокировку сайта.
\end{itemize}

Также важным моментом является то, что закон предусматривает только блокирование всего сайта, а не отдельных страниц. Таким образом возможна ситуация, когда пользователь разместил на сайте запрещенную в Российской Федерации информацию. Владелец сайта по каким то причинам не получил уведомление, и сайт был заблокирован на территории Российской Федерации. В этой ситуации страдают, в первую очередь, обычные пользователи, а владелец сайта может так и не узнать, что его сайт заблокирован.

Редакции 97-ФЗ и 374-ФЗ требуют от организатора распространения информации огромного количества носителей для хранения данных всех пользователей в течении года, а также ущемляют права человека.

Таким образом, большинство из изменений, вносимых с 2012 года больше усложняют всем жизнь, чем выполняют полезную регулятивную функцию.

\section{Список литературы}

[1] Федеральный закон от N 149-ФЗ ФСТЭК России [Электронный ресурс]. — URL: \href{http://fstec.ru/tekhnicheskaya-zashchita-informatsii/dokumenty/107-zakony/364-federalnyj-zakon-ot-27-iyulya-2006-g-n-149-fz}{http://fstec.ru/\linebreak tekhnicheskaya-zashchita-informatsii/dokumenty/107-zakony/364-federalnyj-zakon-ot-27-iyulya-2006-g-n-149-fz} (дата обращения 26.02.2017).

[2] Федеральный закон от N 139-ФЗ КонсультантПлюс [Электронный ресурс]. — URL: \href{http://www.consultant.ru/document/cons_doc_LAW_133282/30b3f8c55f65557c253227a65b908cc075ce114a/#dst100086}{http://www.consultant.\linebreak ru/document/cons\_doc\_LAW\_133282/30b3f8c55f65557c253227a65b908cc075ce114a/\#dst100086} (дата обращения 26.02.2017).

[3] Федеральный закон от N 50-ФЗ КонсультантПлюс [Электронный ресурс]. — URL: \href{http://www.consultant.ru/document/cons_doc_LAW_144630/30b3f8c55f65557c253227a65b908cc075ce114a/#dst100029}{http://www.consultant.\linebreak ru/document/cons\_doc\_LAW\_144630/30b3f8c55f65557c253227a65b908cc075ce114a/\#dst100029} (дата обращения 26.02.2017).

[4] Федеральный закон от N 112-ФЗ КонсультантПлюс [Электронный ресурс]. — URL: \href{http://www.consultant.ru/document/cons_doc_LAW_144630/30b3f8c55f65557c253227a65b908cc075ce114a/#dst100029}{http://www.consultant.\linebreak ru/document/cons\_doc\_LAW\_144630/30b3f8c55f65557c253227a65b908cc075ce114a/\#dst100029} (дата обращения 26.02.2017).

[5] Федеральный закон от N 187-ФЗ КонсультантПлюс [Электронный ресурс]. — URL: \href{http://www.consultant.ru/document/cons_doc_LAW_148497/30b3f8c55f65557c253227a65b908cc075ce114a/#dst100037}{http://www.consultant.\linebreak ru/document/cons\_doc\_LAW\_148497/30b3f8c55f65557c253227a65b908cc075ce114a/\#dst100037} (дата обращения 26.02.2017).

[6] Федеральный закон от N 396-ФЗ КонсультантПлюс [Электронный ресурс]. — URL: \href{http://www.consultant.ru/document/cons_doc_LAW_156535/9fdba7bedb441c57a55c77f449bf400feb99f44b/#dst100312}{http://www.consultant.\linebreak ru/document/cons\_doc\_LAW\_156535/9fdba7bedb441c57a55c77f449bf400feb99f44b/\#dst100312} (дата обращения 26.02.2017).

[7] Федеральный закон от N 398-ФЗ КонсультантПлюс [Электронный ресурс]. — URL: \href{http://www.consultant.ru/document/cons_doc_LAW_156518/3d0cac60971a511280cbba229d9b6329c07731f7/#dst100009}{http://www.consultant.\linebreak ru/document/cons\_doc\_LAW\_156518/3d0cac60971a511280cbba229d9b6329c07731f7/\#dst100009} (дата обращения 26.02.2017).

[8] Федеральный закон от N 97-ФЗ КонсультантПлюс [Электронный ресурс]. — URL: \href{http://www.consultant.ru/document/cons_doc_LAW_162586/3d0cac60971a511280cbba229d9b6329c07731f7/#dst100009}{http://www.consultant.\linebreak ru/document/cons\_doc\_LAW\_162586/3d0cac60971a511280cbba229d9b6329c07731f7/\#dst100009} (дата обращения 26.02.2017).

[9] Федеральный закон от N 222-ФЗ КонсультантПлюс [Электронный ресурс]. — URL: \href{http://www.consultant.ru/document/cons_doc_LAW_165812/30b3f8c55f65557c253227a65b908cc075ce114a/#dst100137}{http://www.consultant.\linebreak ru/document/cons\_doc\_LAW\_165812/30b3f8c55f65557c253227a65b908cc075ce114a/\#dst100137} (дата обращения 26.02.2017).

[10] Федеральный закон от N 242-ФЗ КонсультантПлюс [Электронный ресурс]. — URL: \href{http://www.consultant.ru/document/cons_doc_LAW_165838/3d0cac60971a511280cbba229d9b6329c07731f7/#dst100009}{http://www.consultant.\linebreak ru/document/cons\_doc\_LAW\_165838/3d0cac60971a511280cbba229d9b6329c07731f7/\#dst100009} (дата обращения  26.02.2017).

[11] Федеральный закон от N 364-ФЗ КонсультантПлюс [Электронный ресурс]. — URL: \href{http://www.consultant.ru/document/cons_doc_LAW_171228/3d0cac60971a511280cbba229d9b6329c07731f7/#dst100009}{http://www.consultant.\linebreak ru/document/cons\_doc\_LAW\_171228/3d0cac60971a511280cbba229d9b6329c07731f7/\#dst100009} (дата обращения 26.02.2017).

[12] Федеральный закон от N 531-ФЗ КонсультантПлюс [Электронный ресурс]. — URL: \href{http://www.consultant.ru/document/cons_doc_LAW_173193/3d0cac60971a511280cbba229d9b6329c07731f7/#dst100009}{http://www.consultant.\linebreak ru/document/cons\_doc\_LAW\_173193/3d0cac60971a511280cbba229d9b6329c07731f7/\#dst100009} (дата обращения 26.02.2017).

[13] Федеральный закон от N 188-ФЗ КонсультантПлюс [Электронный ресурс]. — URL: \href{http://www.consultant.ru/document/cons_doc_LAW_181833/3d0cac60971a511280cbba229d9b6329c07731f7/#dst100009}{http://www.consultant.\linebreak ru/document/cons\_doc\_LAW\_181833/3d0cac60971a511280cbba229d9b6329c07731f7/\#dst100009} (дата обращения 26.02.2017).

[14] Федеральный закон от N 263-ФЗ КонсультантПлюс [Электронный ресурс]. — URL: \href{http://www.consultant.ru/document/cons_doc_LAW_182652/30b3f8c55f65557c253227a65b908cc075ce114a/#dst100048}{http://www.consultant.\linebreak ru/document/cons\_doc\_LAW\_182652/30b3f8c55f65557c253227a65b908cc075ce114a/\#dst100048} (дата обращения 26.02.2017).

[15] Федеральный закон от N 264-ФЗ КонсультантПлюс [Электронный ресурс]. — URL: \href{http://www.consultant.ru/document/cons_doc_LAW_182637/3d0cac60971a511280cbba229d9b6329c07731f7/#dst100009}{http://www.consultant.\linebreak ru/document/cons\_doc\_LAW\_182637/3d0cac60971a511280cbba229d9b6329c07731f7/\#dst100009} (дата обращения 26.02.2017).

[16] Федеральный закон от N 208-ФЗ КонсультантПлюс [Электронный ресурс]. — URL: \href{http://www.consultant.ru/document/cons_doc_LAW_200019/3d0cac60971a511280cbba229d9b6329c07731f7/#dst100009}{http://www.consultant.\linebreak ru/document/cons\_doc\_LAW\_200019/3d0cac60971a511280cbba229d9b6329c07731f7/\#dst100009} (дата обращения 26.02.2017).

[17] Федеральный закон от N 374-ФЗ КонсультантПлюс [Электронный ресурс]. — URL: \href{http://www.consultant.ru/document/cons_doc_LAW_201078/4e7c454febb18a75f99a0e0a1256de288dbd7129/#dst100200}{http://www.consultant.\linebreak ru/document/cons\_doc\_LAW\_201078/4e7c454febb18a75f99a0e0a1256de288dbd7129/\#dst100200} (дата обращения 26.02.2017).

[18] Федеральный закон от N 442-ФЗ КонсультантПлюс [Электронный ресурс]. — URL: \href{http://www.consultant.ru/document/cons_doc_LAW_209003/#dst100008}{http://www.consultant.\linebreak ru/document/cons\_doc\_LAW\_209003/\#dst100008} (дата обращения 26.02.2017).

\end{document}