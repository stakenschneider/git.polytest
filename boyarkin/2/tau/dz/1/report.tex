\documentclass[14pt,a4paper,report]{report}
\usepackage[a4paper, mag=1000, left=2.5cm, right=1cm, top=2cm, bottom=2cm, headsep=0.7cm, footskip=1cm]{geometry}
\usepackage[utf8]{inputenc}
\usepackage[english,russian]{babel}
\usepackage{indentfirst}
\usepackage[dvipsnames]{xcolor}
\usepackage[colorlinks]{hyperref}
\usepackage{listings} 
\usepackage{fancyhdr}
\usepackage{caption}
\usepackage{graphicx}
\hypersetup{
	colorlinks = true,
	linkcolor  = black
}

\usepackage{titlesec}
\titleformat{\chapter}
{\Large\bfseries} % format
{}                % label
{0pt}             % sep
{\huge}           % before-code


\DeclareCaptionFont{white}{\color{white}} 

% Listing description
\usepackage{listings} 
\DeclareCaptionFormat{listing}{\colorbox{gray}{\parbox{\textwidth}{#1#2#3}}}
\captionsetup[lstlisting]{format=listing,labelfont=white,textfont=white}
\lstset{ 
	% Listing settings
	inputencoding = utf8,			
	extendedchars = \true, 
	keepspaces = true, 			  	 % Поддержка кириллицы и пробелов в комментариях
	language = C,            	 	 % Язык программирования (для подсветки)
	basicstyle = \small\sffamily, 	 % Размер и начертание шрифта для подсветки кода
	numbers = left,               	 % Где поставить нумерацию строк (слева\справа)
	numberstyle = \tiny,          	 % Размер шрифта для номеров строк
	stepnumber = 1,               	 % Размер шага между двумя номерами строк
	numbersep = 5pt,              	 % Как далеко отстоят номера строк от подсвечиваемого кода
	backgroundcolor = \color{white}, % Цвет фона подсветки - используем \usepackage{color}
	showspaces = false,           	 % Показывать или нет пробелы специальными отступами
	showstringspaces = false,    	 % Показывать или нет пробелы в строках
	showtabs = false,           	 % Показывать или нет табуляцию в строках
	frame = single,              	 % Рисовать рамку вокруг кода
	tabsize = 2,                  	 % Размер табуляции по умолчанию равен 2 пробелам
	captionpos = t,             	 % Позиция заголовка вверху [t] или внизу [b] 
	breaklines = true,           	 % Автоматически переносить строки (да\нет)
	breakatwhitespace = false,   	 % Переносить строки только если есть пробел
	escapeinside = {\%*}{*)}      	 % Если нужно добавить комментарии в коде
}

\begin{document}

\chapter{Домашнее задание №1}

\subsubsection{Бояркин 43501/3}

\section{Основные моменты истории развития теории управления}

Впервые с необходимостью построения регуляторов столкнулись создатели сложных механизмов и высокоточных устройств, в первую очередь – дозаторов, часов.

Во втором веке до нашей эры арабы снабдили поплавковым регулятором уровня водяные часы, чтобы обеспечить постоянную скорость истечения воды. А в 1657 году для похожих целей Гюйгенс встроил в механические часы маятниковый регулятор хода.

Герон Александрийский, живший в первом столетии нашей эры, написал книгу под названием «Пневматика», в которой он привёл несколько чертежей поплавковых регуляторов уровня воды.

Голландский механик и химик К. Дреббель (1572…1633 г.) изобрёл регулятор температуры, который использовал в своих химических опытах и в инкубаторах для выведения цыплят. Этот регулятор содержал устройство, позволяющее выпускать нагретый воздух из камеры, когда температура в ней достигала желаемого результата. Система управления, собранная на основе этого регулятора, считается первой системой с обратной связью, изобретённая в Европе.

Развитие промышленных регуляторов началось лишь на рубеже XVIII и XIX столетий, в эпоху промышленного переворота в Европе. Первыми промышленными регуляторами этого периода являются автоматический поплавковый регулятор питания котла паровой машины, использованный в 1765 году И.И. Ползуновым, и центробежный регулятор скорости паровой машины, на который в 1784 году получил патент Дж. Уатт. Тем самым был открыт фундаментальный принцип управления – принцип обратной связи (принцип Ползунова-Уатта).

Первые публикации по исследованию регуляторов появляются в двадцатых – тридцатых годах XIX века, так Д.С. Чижов опубликовал один из первых трудов в этом направлении в 1823 году.

В 1830 г. Понселе предложил построить регулятор, действие которого было направлено на компенсацию изменения нормального функционирования системы от возмущения. Принцип Понселе (принцип компенсации возмущающего воздействия) – второй фундаментальный принцип управления.

В 1868 г. английский физик Д. Максвелл в работе “О регуляторах” впервые поставил и рассмотрел математическую задачу об устойчивости систем регулирования, где рассмотрены переход к исследованию малых отклонений и линеаризации дифференциальных уравнений, совместное рассмотрение уравнений регулятора и машины, формулировка условий устойчивости линейных систем третьего порядка и постановка перед математиками задачи о нахождении условий устойчивости для уравнений произвольного порядка, в результате чего появилась работа Рауса (критерий Рауса).

В 1876 г. в трудах Парижской академии И.А. Вышнеградский опубликовал статьи “Об общей теории регуляторов” и “О регуляторах прямого действия”. В этих работах содержались не только основные этапы работы Максвелла: системный подход, линеаризация, исследование устойчивости, но и делался существенный шаг вперёд при рассмотрении основных показателей качества процесса регулирования: монотонность, колебательность, апериодичность. Работами И.А. Вышнеградского было вскрыто и объяснено знаменитое противоречие между точностью и устойчивостью регулирования: при уменьшении статической ошибки регулирования ниже некоторого критического значения система теряет устойчивость.

Дальнейшее развитие техники регулирования пошло по пути поиска способов преодоления этого противоречия. Переход от регуляторов прямого действия, перемещающих регулирующие органы непосредственно за счёт энергии измерительного органа, к регуляторам непрямого действия, осуществляющим такие перемещения через силовые усилители, с одной стороны, осложнило проблему устойчивости, введя в контур дополнительные инерционные звенья, с другой стороны, сделало схемы регуляторов более гибкими, дав возможность введения в различные точки схемы дополнительных связей и корректирующих звеньев.

В 1892 г. вышла работа знаменитого русского учёного А.М. Ляпунова ”Общая задача об устойчивости движения”. Теория устойчивости движения, созданная А.М. Ляпуновым, имеет исключительное значение для многих прикладных дисциплин.

Важное место в теории регулирования занимают работы Н.Е. Жуковского «О прочности хода» и «Теория регулирования хода машин» (1909 г.) и работы словацкого учёного А. Стодолы по регулированию гидротурбин.

В период с 1900 по 1940 гг. появляется целый ряд работ, рассматривающих приложения теории регулирования к разнообразным техническим процессам. Особенно чётко мысль о теории регулирования как о дисциплине общетехнического характера проводится в ряде работ И.И. Вознесенского (период с 1922 по 1942 гг.), руководителя одной из крупнейших школ в этой области.

Быстрое развитие систем автоматического управления вело к необходимости создания более эффективных методов исследования. В 1932 г. американский учёный Х. Найквист предложил критерий устойчивости по частотным характеристикам системы в разомкнутом состоянии, а в 1936 г. А.В. Михайлов показывает преимущества применения частотных методов, предложив свой критерий устойчивости, не требующий предварительного размыкания цепи.

С введением частотных методов начинается новый этап ускоренного развития теории управления. Американские учёные Г. Боде и Л. Маккол в 1946 г., русский учёный В.В. Солодовников в 1948 г. разработали метод логарифмических частотных характеристик (ЛЧХ). Если ранее синтез систем осуществлялся путём интуиции и изобретательства, то метод ЛЧХ открыл новые возможности для исследования качества регулирования и создания теории синтеза структур и параметров математическими методами.

В 1940-1950 годы сформировалась по существу новая современная теория автоматического управления. В области устойчивости разработаны методы, существенно облегчающие применение различных критериев устойчивости, введены различные количественные оценки показателей качества процессов регулирования (время регулирования, перерегулирование, колебательность, выброс, степень устойчивости).

В последующие годы в трудах Г.В. Щипанова, В.С. Кулебякина, Б.И. Петрова и других были разработаны теория автоматического регулирования по возмущению и теория компенсации возмущений. В.В. Казаничевым, А.П. Юркевичем, А.А. Фелдбаумом, А.А. Красовским и другими были сформулированы и исследованы принципы экстремального управления и разработана теория экстремальных систем, а также созданы основы теории оптимального управления.

В настоящее время ТАУ представляет собой единую научную базу для решения задач управления объектами различной природы (физической, химической, биологической и т.п.), имея хорошо развитые методы исследования САУ – анализа и синтеза (расчёта и проектирования).

\end{document}