\documentclass[14pt,a4paper,report]{report}
\usepackage[a4paper, mag=1000, left=2.5cm, right=1cm, top=2cm, bottom=2cm, headsep=0.7cm, footskip=1cm]{geometry}
\usepackage[utf8]{inputenc}
\usepackage[english,russian]{babel}
\usepackage{indentfirst}
\usepackage[dvipsnames]{xcolor}
\usepackage[colorlinks]{hyperref}
\usepackage{listings} 
\usepackage{fancyhdr}
\usepackage{caption}
\usepackage{graphicx}
\hypersetup{
	colorlinks = true,
	linkcolor  = black
}

\usepackage{titlesec}
\titleformat{\chapter}
{\Large\bfseries} % format
{}                % label
{0pt}             % sep
{\huge}           % before-code


\DeclareCaptionFont{white}{\color{white}} 

% Listing description
\usepackage{listings} 
\DeclareCaptionFormat{listing}{\colorbox{gray}{\parbox{\textwidth}{#1#2#3}}}
\captionsetup[lstlisting]{format=listing,labelfont=white,textfont=white}
\lstset{ 
	% Listing settings
	inputencoding = utf8,			
	extendedchars = \true, 
	keepspaces = true, 			  	 % Поддержка кириллицы и пробелов в комментариях
	language = C,            	 	 % Язык программирования (для подсветки)
	basicstyle = \small\sffamily, 	 % Размер и начертание шрифта для подсветки кода
	numbers = left,               	 % Где поставить нумерацию строк (слева\справа)
	numberstyle = \tiny,          	 % Размер шрифта для номеров строк
	stepnumber = 1,               	 % Размер шага между двумя номерами строк
	numbersep = 5pt,              	 % Как далеко отстоят номера строк от подсвечиваемого кода
	backgroundcolor = \color{white}, % Цвет фона подсветки - используем \usepackage{color}
	showspaces = false,           	 % Показывать или нет пробелы специальными отступами
	showstringspaces = false,    	 % Показывать или нет пробелы в строках
	showtabs = false,           	 % Показывать или нет табуляцию в строках
	frame = single,              	 % Рисовать рамку вокруг кода
	tabsize = 2,                  	 % Размер табуляции по умолчанию равен 2 пробелам
	captionpos = t,             	 % Позиция заголовка вверху [t] или внизу [b] 
	breaklines = true,           	 % Автоматически переносить строки (да\нет)
	breakatwhitespace = false,   	 % Переносить строки только если есть пробел
	escapeinside = {\%*}{*)}      	 % Если нужно добавить комментарии в коде
}

\begin{document}

\def\contentsname{Содержание}

% Titlepage
\begin{titlepage}
	\begin{center}
		\textsc{Санкт-Петербургский Политехнический 
			Университет Петра Великого\\[5mm]
			Кафедра компьютерных систем и программных технологий}
		
		\vfill
		
		\textbf{Отчёт по лабораторной работе №1\\
		"Определение векторов смещения"\\[3mm]
			Курс: «Разработка графических приложений»\\[41mm]
		}
	\end{center}
	
	\hfill
	\begin{minipage}{.4\textwidth}
		Выполнил студент:\\[2mm] 
		Волкова М.Д.\\
		Группа: 13541/2\\[5mm]
		
		Проверил:\\[2mm] 
		Абрамов Н.А.
	\end{minipage}
	\vfill
	\begin{center}
		Санкт-Петербург\\ \the\year\ г.
	\end{center}
\end{titlepage}

% Contents
\tableofcontents
\clearpage

\chapter{Лабораторная работа №1}

\section{Цель работы}

Реализовать алгоритм поиска векторов смещения на изображения, используя метрику SAD. 

\section{Описание алгоритма поиска смещения на изображении}

\begin{enumerate}
\item Выбор двух последовательных изображений
\item Разбиение изображения на блоки 8x8 пикселей
\item Поиск похожих блоков первого изображения во втором с помощью метрики SAD в окрестности 16х16 пикселей методом полного перебора
\item Визуализация векторов смещения, наложенных на исходное изображение
\end{enumerate}

\section{Эксперименты}
\begin{enumerate}
	\item Нахождение и отображение векторов смещения на изображениях:
	\item Два последовательных кадра из нескольких роликов
	\item Смещение изображение вправо 
\end{enumerate}

\clearpage

\section{Результаты}
Рассмотрим эксперимент, в котором будем определять вектора смещений для 2 последовательных изображений. 

\section{Вывод}
С помощью данного алгоритма возможно качественно определить вектора смещения на изображении. 

Однако на этапе выбора материалов для проведения экспериментов, было выявлено, что на однородных участках изображений однозначно определить вектор смещения не удается. 

Также следует отметить, что и после выбора материалов экспериментов однородность изображения сказывалась на результате, в каждом из продемонстрированных можно увидеть недостатки связанные с однородностью изображения. Тем не менее не следует признавать данные недостатки критическими.

Неприемлемые результаты расчета векторов смещения объясняются тем, что возможно исчезновение частей изображения от кадра к кадру, именно под эту ситуацию подходят в частности границы. 
	
\chapter{Листинг}
	
\lstinputlisting{code/main.cpp}

\end{document}