
% Быть посвободнее при склеивании слов
\sloppy

% Настройка листингов
\renewcommand{\lstlistingname}{Листинг}
\lstset{
	frame=single, % adds a frame around the code
	rulesepcolor=\color{gray},
	rulecolor=\color{black},
	breaklines=true,
	xleftmargin=2em,
	extendedchars={true},
	inputencoding={utf8},
	basicstyle={\ttfamily \scriptsize},
	keywordstyle={\rmfamily \bfseries},
	commentstyle={\rmfamily \itshape},
	tabsize={2},
	numbers={left},
	frame={single},
	showstringspaces={false},
}
\lstdefinestyle{java}{
	breaklines={true},
	texcl=true,
	language={Java},
}
\lstset{
    literate={а}{{\selectfont\char224}}1
    {б}{{\selectfont\char225}}1
    {в}{{\selectfont\char226}}1
    {г}{{\selectfont\char227}}1
    {д}{{\selectfont\char228}}1
    {е}{{\selectfont\char229}}1
    {ё}{{\"e}}1
    {ж}{{\selectfont\char230}}1
    {з}{{\selectfont\char231}}1
    {и}{{\selectfont\char232}}1
    {й}{{\selectfont\char233}}1
    {к}{{\selectfont\char234}}1
    {л}{{\selectfont\char235}}1
    {м}{{\selectfont\char236}}1
    {н}{{\selectfont\char237}}1
    {о}{{\selectfont\char238}}1
    {п}{{\selectfont\char239}}1
    {р}{{\selectfont\char240}}1
    {с}{{\selectfont\char241}}1
    {т}{{\selectfont\char242}}1
    {у}{{\selectfont\char243}}1
    {ф}{{\selectfont\char244}}1
    {х}{{\selectfont\char245}}1
    {ц}{{\selectfont\char246}}1
    {ч}{{\selectfont\char247}}1
    {ш}{{\selectfont\char248}}1
    {щ}{{\selectfont\char249}}1
    {ъ}{{\selectfont\char250}}1
    {ы}{{\selectfont\char251}}1
    {ь}{{\selectfont\char252}}1
    {э}{{\selectfont\char253}}1
    {ю}{{\selectfont\char254}}1
    {я}{{\selectfont\char255}}1
    {А}{{\selectfont\char192}}1
    {Б}{{\selectfont\char193}}1
    {В}{{\selectfont\char194}}1
    {Г}{{\selectfont\char195}}1
    {Д}{{\selectfont\char196}}1
    {Е}{{\selectfont\char197}}1
    {Ё}{{\"E}}1
    {Ж}{{\selectfont\char198}}1
    {З}{{\selectfont\char199}}1
    {И}{{\selectfont\char200}}1
    {Й}{{\selectfont\char201}}1
    {К}{{\selectfont\char202}}1
    {Л}{{\selectfont\char203}}1
    {М}{{\selectfont\char204}}1
    {Н}{{\selectfont\char205}}1
    {О}{{\selectfont\char206}}1
    {П}{{\selectfont\char207}}1
    {Р}{{\selectfont\char208}}1
    {С}{{\selectfont\char209}}1
    {Т}{{\selectfont\char210}}1
    {У}{{\selectfont\char211}}1
    {Ф}{{\selectfont\char212}}1
    {Х}{{\selectfont\char213}}1
    {Ц}{{\selectfont\char214}}1
    {Ч}{{\selectfont\char215}}1
    {Ш}{{\selectfont\char216}}1
    {Щ}{{\selectfont\char217}}1
    {Ъ}{{\selectfont\char218}}1
    {Ы}{{\selectfont\char219}}1
    {Ь}{{\selectfont\char220}}1
    {Э}{{\selectfont\char221}}1
    {Ю}{{\selectfont\char222}}1
    {Я}{{\selectfont\char223}}1
}


% Настройка стиля оглавления
% \renewcommand{\tocchapterfont}{}

%%%%%%%%%%%%%%%%%%%%%%%%%%%%%%%%%%%%%%%%%%%%%%%%%%%%%%%%%%%%%%%%%%%%%%%%%%%%%%%%


\begin{document}

% Заведующий кафедрой
\apname{В.М.~Ицыксон}

% Название
\title{ВЫПУСКНАЯ РАБОТА БАКАЛАВРА}

% Тема
\topic{Разработка учебно-методических средств для исследования моделей глубокого обучения}

% Направление
\coursenum{09.03.01}
\course{Информатика и вычислительная техника}
\masterprognum{09.03.01\_15}
\masterprog{Технологии проектирования системного и прикладного программного обеспечения}

% Автор
\author{Волкова М.Д.}
\group{43501/3}

% Научный руководитель
\sa{Никитин К.В.}
\sastatus{к.~т.~н.,~доц.}

% Рецензент
% \rev{Р.Е.~Цензент}
% \revstatus{к.~т.~н.,~доц.}

% Консультант
%\conspec{нормоконтролю}
%\con{А.Г.~Новопашенный}
%\constatus{к.т.н., доцент}

% Уменьшить размер шрифта для названия института, так как он не влезает в
% одну строчку по новому размеру страницы
\renewcommand\instfont{\small}
% Переопределение названий Университета/Факультета/Кафедры
%\institution{Усть-Гатчинский государственный университет кирпично-велосипедной промышленности}
%\faculty{Институт кройки и шитья}
%\department{Кафедра построения конструкций из пластилина}

\logo{fig/spbpu.jpg}

\def\contentsname{Содержание}


% Contents
%\tableofcontents
%\clearpage


\section{Цель работы}

Анализ свойств заданного объекта. Построение матриц для перехода между формами представления ВСВ.

\section{Программа работы}

\begin{enumerate}
	\item Проанализировать свойства заданного объекта:
\begin{itemize}
\item управляемость
\item наблюдаемость
\item устойчивость
\item минимальность
\item минимально фазовость
\end{itemize}
\item Рассчитать матрицы перехода

\end{enumerate}

\section{Индивидуальное задание}

$x''+2x'+0.75x=0.75u$, \\

\section{Ход работы}
\subsection{Свойства объекта}

\subsubsection{Управляемость}

Проверим управляемость системы по критерию Калмана:

\begin{equation*}
\text{$A=\begin{bmatrix}
0 & 1 \\ -0.75 & -2
\end{bmatrix} B=\begin{bmatrix}
0 \\ 1
\end{bmatrix}$} \quad
\text{$
U = \begin{bmatrix}
B & AB
\end{bmatrix} = \begin{bmatrix}
0 & 1\\ 1 & -2
\end{bmatrix} 
$} \quad
\text{$detU=-1\neq 0$}
\end{equation*}

Определитель одной из матриц управляемости не нулевой, что означает, что система полностью управляема.

\subsubsection{Наблюдаемость}

Проверим наблюдаемость системы по критерию Калмана:

\begin{equation*}
\text{$A=\begin{bmatrix}
0 & -0.75 \\ 1 & -2
\end{bmatrix} C=\begin{bmatrix}
0 & 1
\end{bmatrix}$} \quad
\end{equation*}

\begin{equation*}
\text{$N=[C^T,A^TC^T]=[\begin{bmatrix} 0 \\ 1 \end{bmatrix},\begin{bmatrix}   0  &  1 \\
   -0.75 &  -2 \end{bmatrix}\begin{bmatrix} 0 \\ 1 \end{bmatrix}]=\begin{bmatrix} 0 & 1 \\ 1 & -2 \end{bmatrix}$}
\end{equation*}

\begin{equation*}
\text{$detN=-1\neq 0$}
\end{equation*}

Определитель одной из матриц наблюдаемости не нулевой, что означает, что система полностью наблюдаема.

\subsubsection{Устойчивость}

По теореме Ляпунова система является устойчивой тогда, когда вещественные части полюсов её передаточной функции отрицательны. В нашем случае полюса передаточной функции равны $p_1=-1.5, p_2=-0.5$, что означает, что система устойчива, так как все корни лежат в левой полуплоскости. 

\subsubsection{Минимальность}

Система минимальна, так как её порядок понизить нельзя.

\subsection{Минимально фазовость}

Звено является минимально фазовым, так как все корни лежат в левой полуплоскости. 

\subsection{Преобразования форм}

\subsubsection{Матрицы управляемости}

Матрица управляемости находится как блочная матрица, где первый столбец равен матрице $B$, а второй столбец равен произведению $AB$:

\begin{center}
$U=[B, AB]$
\end{center}

Матрицы управляемости нормальной формы управления (НФУ):

\begin{equation*}
\text{$U= \begin{bmatrix}
0 & 1\\ 1 & -2
\end{bmatrix}$}
\text{$, U^{-1}=\begin{bmatrix} 2 & 1 \\ 1 & 0 \end{bmatrix}$}
\end{equation*}

Матрицы управляемости нормальной формы наблюдения (НФН):

\begin{equation*}
\text{$U=\begin{bmatrix} 0.75 &         0 \\
         0  &  0.75 \end{bmatrix}$}
\quad
\text{$U^{-1}=\begin{bmatrix} 1.3333   &      0 \\
         0  &  1.3333 \end{bmatrix}$}
\end{equation*}

Матрицы управляемости канонической формы (КФ):

\begin{equation*}
\text{$U=\begin{bmatrix}  -0.75  &  1.125 \\
    0.75  & -0.375 \end{bmatrix}$}
\quad
\text{$U^{-1}=\begin{bmatrix} 0.6667 &   2 \\
    1.3333  &  1.3333 \end{bmatrix}$}
\end{equation*}

\subsubsection{Матрицы преобразования}

Матрица преобразования высчитывается по формуле:

\begin{center}
$P=U_{*}U^{-1}$
\end{center}

\begin{itemize}
	\item Матрица преобразования из НФУ
	\begin{itemize}
	\item НФУ в НФН
	\begin{equation*}
	\text{$P=U_{*}U^{-1}=
	\begin{bmatrix} 0.75 &         0 \\
         0  &  0.75 \end{bmatrix}
	\begin{bmatrix} 2 & 1 \\ 1 & 0 \end{bmatrix}
	=\begin{bmatrix}1.5 &   0.75 \\
    0.75 &         0 \end{bmatrix}$}
	\end{equation*}
	
	Проверим корректность полученной матрицы преобразования $P$. Для этого получим матрицу $B_{*}$ через матрицу $B$. 
	
	\begin{equation*}
	\text{$B_{*}=PB$}
	\Longrightarrow
	\text{$B_{*}
	=\begin{bmatrix}1.5 &   0.75 \\
    0.75 &         0 \end{bmatrix}
	\begin{bmatrix} 0 \\ 1 \end{bmatrix}
	=\begin{bmatrix} 0.75 \\ 0 \end{bmatrix}$}
	\end{equation*}
	
	\item НФУ в КФ:
	
	\begin{equation*}
	\text{$P=U_{*}U^{-1}=
	\begin{bmatrix}  -0.75  &  1.125 \\
    0.75  & -0.375 \end{bmatrix}
	\begin{bmatrix} 2 & 1 \\ 1 & 0 \end{bmatrix}
	=\begin{bmatrix} -0.375 &   -0.75 \\
    1.125 &    0.75 \end{bmatrix}$}
	\end{equation*}
	
	Проверим корректность полученной матрицы преобразования $P$. Для этого получим матрицу $B_{*}$ через матрицу $B$. 
	
	\begin{equation*}
	\text{$B_{*}=PB$}
	\Longrightarrow
	\text{$B_{*}=
	\begin{bmatrix} -0.375 &   -0.75 \\
    1.125 &    0.75 \end{bmatrix}
	\begin{bmatrix} 0 \\ 1 \end{bmatrix}
	=\begin{bmatrix} -0.75 \\ 0.75 \end{bmatrix}$}
	\end{equation*}
	
	\end{itemize}
	\item Матрица преобразования из НФН:
	\begin{itemize}

	\item НФН в НФУ
		
	\begin{equation*}
	\text{$P=U_{*}U^{-1}=
	\begin{bmatrix}
0 & 1\\ 1 & -2
\end{bmatrix}
\begin{bmatrix} 1.3333   &      0 \\
         0  &  1.3333 \end{bmatrix}
=\begin{bmatrix}   0  &  1.3333 \\
    1.3333  & -2.6667 \end{bmatrix}$}
	\end{equation*}
	
	Проверим корректность полученной матрицы преобразования $P$. Для этого получим матрицу $B_{*}$ через матрицу $B$.
	
	\begin{equation*}
	\text{$B_{*}=PB$}
	\Longrightarrow
	\text{$B_{*}=
	\begin{bmatrix}   0  &  1.3333 \\
    1.3333  & -2.6667 \end{bmatrix}
   \begin{bmatrix} 0.75 \\ 0 \end{bmatrix}
   =\begin{bmatrix} 0 \\ 1 \end{bmatrix}$}
	\end{equation*}
	
	
	
	\item НФН в КФ:
	
	\begin{equation*}
	\text{$P=U_{*}U^{-1}=
	\begin{bmatrix}  -0.75  &  1.125 \\
    0.75  & -0.375 \end{bmatrix}
\begin{bmatrix} 1.3333   &      0 \\
         0  &  1.3333 \end{bmatrix}
=\begin{bmatrix}   -1 &    1.5 \\
    1  & -0.5 \end{bmatrix}$}
	\end{equation*}
	
	Проверим корректность полученной матрицы преобразования $P$. Для этого получим матрицу $B_{*}$ через матрицу $B$.
	
	\begin{equation*}
	\text{$B_{*}=PB$}
	\Longrightarrow
	\text{$B_{*}=
	\begin{bmatrix}   -1 &    1.5 \\
    1  & -0.5 \end{bmatrix}
\begin{bmatrix} 0.75 \\ 0 \end{bmatrix}
   =\begin{bmatrix} -0.75 \\ 0.75 \end{bmatrix}$}
	\end{equation*}
	

	\end{itemize}	
	
	\item Матрица преобразования из КФ:
	\begin{itemize}
	\item КФ в НФУ

\begin{equation*}
	\text{$P=U_{*}U^{-1}=
	\begin{bmatrix}
0 & 1\\ 1 & -2
\end{bmatrix}
\begin{bmatrix} 0.6667 &   2 \\
    1.3333  &  1.3333 \end{bmatrix}
=\begin{bmatrix}1.3333  &  1.3333 \\
   -2  & -0.6667\end{bmatrix}$}
	\end{equation*}
	
	Проверим корректность полученной матрицы преобразования $P$. Для этого получим матрицу $B_{*}$ через матрицу $B$.
	
	\begin{equation*}
	\text{$B_{*}=PB$}
	\Longrightarrow
	\text{$B_{*}=
	\begin{bmatrix}1.3333  &  1.3333 \\
   -2  & -0.6667\end{bmatrix}
   \begin{bmatrix} -0.75 \\ 0.75 \end{bmatrix}
   =\begin{bmatrix} 0 \\ 1 \end{bmatrix}$}
	\end{equation*}
	

	
	\item КФ в НФН:
	
	\begin{equation*}
	\text{$P=U_{*}U^{-1}=
	\begin{bmatrix} 0.75 &         0 \\
         0  &  0.75 \end{bmatrix}
	\begin{bmatrix} 0.6667 &   2 \\
    1.3333  &  1.3333 \end{bmatrix}
=\begin{bmatrix}0.5 &  1.5 \\
    1  &  1\end{bmatrix}$}
	\end{equation*}
	
	Проверим корректность полученной матрицы преобразования $P$. Для этого получим матрицу $B_{*}$ через матрицу $B$.
	
	\begin{equation*}
	\text{$B_{*}=PB$}
	\Longrightarrow
	\text{$B_{*}=
\begin{bmatrix}0.5 &  1.5 \\
    1  &  1\end{bmatrix}
         \begin{bmatrix} -0.75 \\ 0.75 \end{bmatrix}
         =\begin{bmatrix} 0.75 \\ 0 \end{bmatrix}$}
	\end{equation*}
		
	\end{itemize}
\end{itemize}


\section{Вывод}

В ходе работы проанализированы свойства объекта и получены матрицы для переходов между
формами представления модели ВСВ. Также проверена корректность всех матриц перехода.
 
\end{document}